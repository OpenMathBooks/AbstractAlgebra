\documentclass[12pt]{exam}

\usepackage{amsmath, amssymb, amsthm, multicol}
\usepackage{graphicx}
\usepackage{textcomp}

\def\d{\displaystyle}
\def\matrix#1{\begin{bmatrix}#1\end{bmatrix}}
\def\b{\mathbf}
\def\R{\mathbb{R}}
\def\Z{\mathbb{Z}}
\def\Q{\mathbb{Q}}
\def\and{\wedge}
\def\imp{\rightarrow}
\def\inv{^{-1}}



%\pointname{pts}
\pointsinmargin
\marginpointname{pts}
\addpoints
\pagestyle{head}
%\printanswers

\firstpageheader{Math 322}{\bf (Take Home) Quiz 1 }{Due Wed, January 30}


\begin{document}

%space for name
 \noindent {\large\bf Name:} \underline{\hspace{2.5in}}
 \vskip 1em

\begin{questions}
\question[10] Consider the polynomial $x^3 - 7$ in $\Q[x]$.  Outside of $\Q$, this has a root $\sqrt[3]{7}$, so let's consider the extension field $\Q(\sqrt[3]{7})$.  We can also consider the quotient ring $\Q[x]/\langle x^3 - 7\rangle$.
\begin{parts}
\part In $\Q[x]/\langle x^3 - 7\rangle$, find the inverse of the coset $\langle x^3 - 7\rangle + x^2 - x$.  You should use the Euclidean algorithm (show all your steps) and explain why your answer is correct.
\begin{solution}
We will have $\langle x^3 - 7\rangle + x-2$ as the inverse.  If we apply the Euclidean algorithm with $x^3 - 7$ and $x^2 + 2x + 4$, we get

\end{solution}
\vfill
\vfill
\vfill
\part Use your solution above to find the inverse of the element $-\sqrt[3]{7} + \sqrt[3]{7}^2$ in $\Q(\sqrt[3]{7})$.  Briefly explain how you know your answer must be correct, even if you don't do any calculations in the field.
\begin{solution}

\end{solution}
\vfill
\end{parts}
\end{questions}
\end{document}


