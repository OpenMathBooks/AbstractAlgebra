\documentclass[12pt]{exam}

\usepackage{amsmath, amssymb, multicol}
\usepackage{graphicx}
\usepackage{textcomp}

\def\d{\displaystyle}
\def\matrix#1{\begin{bmatrix}#1\end{bmatrix}}
\def\b{\mathbf}
\def\R{\mathbb{R}}
\def\Z{\mathbb{Z}}
\def\and{\wedge}
\def\imp{\rightarrow}
\def\inv{^{-1}}



%\pointname{pts}
\pointsinmargin
\marginpointname{pts}
\addpoints
\pagestyle{head}
%\printanswers

\firstpageheader{Math 321}{\bf Quiz 6}{Friday, March 29}


\begin{document}

%space for name
\noindent {\large\bf Name:} \underline{\hspace{2.5in}}
\vskip 1em

\begin{questions}
\question Consider the group $\Z_{18}$
\begin{parts}
\part Write a normal series for $\Z_{18}$ of length at least two that is NOT a composition series.
\begin{solution}
You have a few choices:
\[\Z_{18} \supset \langle 3\rangle \supset \{0\}\]

\[\Z_{18} \supset \langle 9\rangle \supset \{0\}\]
\[\Z_{18} \supset \langle 6\rangle \supset \{0\}\]
\[\Z_{18} \supset \langle 2\rangle \supset \{0\}\]

\end{solution}
\vfill
\part Show how you can use quotient groups to find a refinement of the series that \emph{is} a composition series.
\begin{solution}
For example, if you started with 
\[\Z_{18} \supset \langle 6\rangle \supset \{0\}\]
the quotient groups will be 
\[\Z_{18}/\langle 6 \rangle \cong \Z_6\]
\[\langle 6 \rangle /\{0\} \cong \Z_3. \]
We can split the first one since $\Z_6$ is not simple.  One of the subgroup of $\Z_6$ is $\{0, 3\}\cong \Z_{2}$, and this corresponds to $\{0+\langle 6\rangle, 3+\langle 6 \rangle\}$.  Those are exactly the cosets you get in the quotient group $\{\langle 3\rangle/\langle 6 \rangle\}$.  So the intermediate subgroup is $\langle 3 \rangle$ giving the composition series 
\[\Z_{18} \supset \langle 3 \rangle \supset \langle 6\rangle \supset \{0\}\]
\end{solution}
\vfill
\vfill
\part Write down a second composition series for the group and find all of its quotient groups.  How do these quotient groups compare to the quotient groups for the composition series you found in part (b)?
\begin{solution}
  There are three:
  \[\Z_{18} \supset \langle 2 \rangle \supset \langle 6\rangle \supset \{0\}\]
  \[\Z_{18} \supset \langle 3 \rangle \supset \langle 6\rangle \supset \{0\}\]
  \[\Z_{18} \supset \langle 3 \rangle \supset \langle 9 \rangle \supset \{0\}\]
The corresponding sequences of quotient groups are:
\[\Z_2, \qquad \Z_3, \qquad \Z_3\]
\[\Z_3, \qquad \Z_2, \qquad \Z_3\]
\[\Z_3, \qquad \Z_3, \qquad \Z_2.\]
These are the same, just in different orders.
\end{solution}
\vfill
\end{parts}


\end{questions}
\end{document}


