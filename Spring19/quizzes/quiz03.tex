 \documentclass[12pt]{exam}

\usepackage{amsmath, amssymb, amsthm, multicol}
\usepackage{graphicx}
\usepackage{textcomp}

\def\d{\displaystyle}
\def\matrix#1{\begin{bmatrix}#1\end{bmatrix}}
\def\b{\mathbf}
\def\R{\mathbb{R}}
\def\Z{\mathbb{Z}}
\def\Q{\mathbb{Q}}
\def\and{\wedge}
\def\imp{\rightarrow}
\def\inv{^{-1}}
\DeclareMathOperator{\ord}{ord}
\def\st{:}
\def\onto{\twoheadrightarrow}
\def\Gal{\mathrm{Gal}}



%\pointname{pts}
\pointsinmargin
\marginpointname{pts}
\addpoints
\pagestyle{head}
%\printanswers

\firstpageheader{MATH 322}{\bf Quiz 3}{Wednesday, February 13}


\begin{document}

%space for name
\noindent {\large\bf Name:} \underline{\hspace{2.5in}}
\vskip 1em

\begin{questions}
\question Find the minimal polynomial for $\frac{1+\sqrt{3}}{5}$ in $\Q[x]$.
\begin{solution}
  We have $x = \frac{1+\sqrt{3}}{5}$ so $5x - 1 = \sqrt{3}$ which gives $25x^2 -10 x + 1 = 3$.  Thus a polynomial that has the number as a root is 
  \[25x^2 - 10x -2\]
  This polynomial must be irreducible since $\frac{1+\sqrt{3}}{5} \notin\Q$ (or by Eisenstien), which makes it the minimal polynomial.
\end{solution}
\vfill

\question Is $\frac{1+\sqrt{3}}{5} \in \Q(\sqrt{3})$?  Briefly explain.

\begin{solution}
  Yes.  Because $\Q(\sqrt{3})$ is a field, it is closed under field operations.
\end{solution}

\vfill
\question Is $\Q(\sqrt{3})$ the splitting field for the polynomial you found in question 1?  Briefly explain.

\begin{solution}
  Yes, it must be.  Certainly $\Q(\frac{1+\sqrt{3}}{5})$ is the splitting field, since the polynomial will factor completely in it.  But this is the same field as $\Q(\sqrt{3})$.
\end{solution}

\vfill

\question Is there an automorphism of $\Q(\sqrt{3})$ which sends $\sqrt{3}$ to $\frac{1+\sqrt{3}}{5}$?  Explain how you know using polynomials.

\begin{solution}
  No there is not, since these are not the roots of the same irreducible polynomial.
\end{solution}

\vfill
\question Find a non-identity element of the Galois group $\Gal(\Q(\sqrt{3})/\Q)$, and say what it does to $\frac{1+\sqrt{3}}{5}$.
\begin{solution}
  We must send $\sqrt{3}$ to another root of \emph{its} minimal polynomial $x^2 - 3$.  So $\sigma(\sqrt{3}) = -\sqrt{3}$.  Then we know that $\sigma(\frac{1+\sqrt{3}}{5}) = \frac{1-\sqrt{3}}{5}$.  
  
  Note that this tells us that $\frac{1-\sqrt{3}}{5}$ is the other root of the polynomial from question 1.
\end{solution}
\vfill
\end{questions}
\end{document}


