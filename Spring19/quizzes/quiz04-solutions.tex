 \documentclass[12pt]{exam}

\usepackage{amsmath, amssymb, amsthm, multicol}
\usepackage{graphicx}
\usepackage{textcomp}

\def\d{\displaystyle}
\def\matrix#1{\begin{bmatrix}#1\end{bmatrix}}
\def\b{\mathbf}
\def\R{\mathbb{R}}
\def\Z{\mathbb{Z}}
\def\Q{\mathbb{Q}}
\def\and{\wedge}
\def\imp{\rightarrow}
\def\inv{^{-1}}
\DeclareMathOperator{\ord}{ord}
\def\st{:}
\def\onto{\twoheadrightarrow}
\def\Gal{\mathrm{Gal}}



%\pointname{pts}
\pointsinmargin
\marginpointname{pts}
\marginbonuspointname{bn-pts}
\addpoints
\pagestyle{head}
\printanswers

\firstpageheader{MATH 322}{\bf Quiz 4 Solutions}{Monday, February 25}


\begin{document}

%space for name
\noindent {\large\bf Name:} \underline{\hspace{2.5in}}
\vskip 1em

\begin{questions}
  \question Let $G$ be a group with elements $\{e, a, b, c, d, f, g, h\}$.  Here is part of the table for $G$:
  \begin{center}
    \begin{tabular}{c|cccccccc}
           & $e$ & $a$ & $b$ & $c$ & $d$ & $f$ & $g$ & $h$ \\ \hline
       $\vdots$ & &&& \vdots & & & & \\
       $c$ & $c$ & $b$ & $e$ & $a$ & $h$ & $g$ & $d$ & $f$ \\ 
       $\vdots$ & &&& \vdots & & & & 
     \end{tabular}
  \end{center}
  \begin{parts}
    \part[3] Find the left regular representation of $c$ as in Cayley's theorem.  That is, find the permutation $\lambda_c$ in $S_8$ corresponding to $c$.
    \begin{solution}
     $\lambda_c = \begin{pmatrix} 1 & 2 & 3 & 4 & 5 & 6 & 7 & 8 \\ 4 & 3 & 1 & 2 & 8 & 7 & 5 & 6\end{pmatrix}$
    \end{solution}

    \vfill
    \part[3] Suppose $\lambda_d = \begin{pmatrix}1 & 2 & 3 & 4 & 5 & 6 & 7 & 8\\ 5 & 6 & 7 & 8 & 2 & 1 & 3 & 4\end{pmatrix}$ is the left regular representation of $d$.  Write the row for $d$ in the group table. 
    \begin{solution}
     The row will read $d~f~g~h~a~e~b~c$
    \end{solution}

    \vfill
    \part[4] If you compute $\lambda_c\lambda_d$ you get another permutation in $S_8$.  Of which element in $G$ is it the left regular representation?  That is, find $x \in G$ such that $\lambda_x = \lambda_c\lambda_d$.  Briefly explain how you know you are right.
    \begin{solution}
     $\lambda_c\lambda_d = \lambda_{cd}$ by the way we define regular representations.  But because $G$ is isomorphic to the group of regular representations, we know $\lambda_{cd}$ corresponds to $cd$ which in $G$ equals $h$.  Thus $\lambda_c\lambda_d = \lambda_h$.
    \end{solution}

    \vfill
    \vfill
  \end{parts}
  
  \bonusquestion[3] Bonus: Which of the permutations you found above are even?  Briefly explain how you know
  \vskip 1em
\end{questions}
\end{document}


