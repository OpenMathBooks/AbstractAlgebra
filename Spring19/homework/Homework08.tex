\documentclass[11pt]{exam}

\usepackage{amsmath, amssymb, amsthm, multicol}
\usepackage{graphicx}
\usepackage{textcomp}
\usepackage{tikz}
\usepackage{mathrsfs}
%\usepackage[top=1in, bottom=1in, left=1in, right=1in]{geometry}
%\usepackage{framed}


%\newenvironment{ques}[2]{\vskip 1ex \noindent{\bf Question #1:}\marginpar{(#2 pts)}}{}
%\newenvironment{sol}{\begin{framed} \noindent{\em Solution:}}{\end{framed}\vskip 1em}



\def\d{\displaystyle}
\def\b{\mathbf}
\def\N{\mathbb{N}}
\def\R{\mathbb{R}}
\def\Z{\mathbb{Z}}
\def\Q{\mathbb{Q}}
\def\C{\mathbb{C}}
\def\F{\mathscr{F}}
\def\st{~:~}
\def\bar{\overline}
\def\inv{^{-1}}
\def\imp{\rightarrow}
\def\and{\wedge}
\def\onto{\twoheadrightarrow}
\DeclareMathOperator{\Gal}{Gal}


%\pointname{pts}
\pointsinmargin
\marginpointname{pts}
\marginbonuspointname{bns-pts}
\addpoints
\pagestyle{head}
%\printanswers

\firstpageheader{Math 322}{\bf Homework 8}{Due: Friday, May 3}




\begin{document}

\noindent \textbf{Instructions}: Same rules as always.  Turn this in early for a 5\% bonus for every day it is early (before Friday, May 3rd).
\vskip 1ex
\begin{questions}

% %or fewer points?
% \question[3] Recall that we defined the {\em action} of a group $G$ on a set $X$ to be a function from $G\times X$ to $X$ written $(g,x) \mapsto gx$ which satisfies (a) $ex = x$ and (b) $(g_1g_2)x = g_1(g_2x)$.  We said two elements $x,y \in X$ were {\em $G$-equivalent} if there was some $g \in G$ such that $gx = y$, and we write $x \sim y$.  Prove that $\sim$ is an equivalence relation.


\question[6] Let $G$ be the additive group of real numbers, $(\R, +)$ and let $X = \R^2$ be the real plane.  For each point $P \in X$ and $r \in G$, define $(r,P)$ to be the point $P$ lands on after rotating the plane counterclockwise about the origin through $r$ radians.
\begin{parts}
\part Prove that $X$ is a $G$-set.  That is, prove that the procedure above defines an action of $G$ on $X$.
\part Pick a point $P$ other than the origin.  Describe geometrically the orbit containing $P$.
\part Find the group $G_P$. 
\end{parts}


\question[9] Let $X = \{1,2,3,4,5,6\}$ and let $G = \{(1), (123), (132), (45), (123)(45), (132)(45)\}$.  Let $G$ act on $X$ in the obvious way.
\begin{parts}
\part For each $x \in X$ and $g \in G$, find $\mathcal{O}_x$, $G_x$ and $X_g$.  Label these clearly.
\part Verify the \emph{orbit-stabilizer theorem} and \emph{Burnside's lemma} for this example and explain (i.e., demonstrate that you know what these are and mean).
\part To thank your professors for doing such an amazing job all semester, you decide to bake 6 pies.  You will give 3 to your favorite abstract algebra teacher, 2 to your next favorite teacher, and 1 to your third favorite.  You know how to make 4 different types of pies.  How many different pie-to-professor combinations can you create?  Use Burnside's lemma and explain how this relates to the $X$ and $G$ in this problem.
\end{parts}




\question[12] There are 24 different rotations of a cube (think about where the axis of rotation could be).  Use these, together with Burnside's lemma to count the number of ways to color the cube as described below.  In each case, explain how your arrived at your answer (make a table relating the sorts of rotations to size of their fixed sets).
\begin{parts}
\part Describe the 24 rotations carefully (break them into groups).
\part Color the faces of a cube using up to 4 colors.
\part Color the vertices of a cube using up to 3 colors.
\part Color the edges of a  cube using up to 3 colors.
\end{parts}


\question[3] How many ways can you label a 6-sided die (number cube) with the numbers 1 through 6, giving each side a different number?



\end{questions}

\end{document}


