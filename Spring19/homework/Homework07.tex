\documentclass[11pt]{exam}

\usepackage{amsmath, amssymb, amsthm, multicol}
\usepackage{graphicx}
\usepackage{textcomp}
\usepackage{tikz}
\usepackage{mathrsfs}
%\usepackage[top=1in, bottom=1in, left=1in, right=1in]{geometry}
%\usepackage{framed}


%\newenvironment{ques}[2]{\vskip 1ex \noindent{\bf Question #1:}\marginpar{(#2 pts)}}{}
%\newenvironment{sol}{\begin{framed} \noindent{\em Solution:}}{\end{framed}\vskip 1em}



\def\d{\displaystyle}
\def\b{\mathbf}
\def\N{\mathbb{N}}
\def\R{\mathbb{R}}
\def\Z{\mathbb{Z}}
\def\Q{\mathbb{Q}}
\def\C{\mathbb{C}}
\def\F{\mathscr{F}}
\def\st{~:~}
\def\bar{\overline}
\def\inv{^{-1}}
\def\imp{\rightarrow}
\def\and{\wedge}
\def\onto{\twoheadrightarrow}
\DeclareMathOperator{\Gal}{Gal}
\DeclareMathOperator{\ord}{ord}
\DeclareMathOperator{\lcm}{lcm}

%\pointname{pts}
\pointsinmargin
\marginpointname{pts}
\marginbonuspointname{bns-pts}
\addpoints
\pagestyle{head}
%\printanswers

\firstpageheader{Math 322}{\bf Homework 7}{Due: Wednesday, April 10}




\begin{document}

\noindent \textbf{Instructions}: Same rules as usual.  Work together, write-up alone, no internet!
\vskip 1ex
\begin{questions}

\question[9] Consider the normal series below for $\Z_{24}$:
\[\Z_{24} \supset \langle 12\rangle \supset \{0\}\]
\begin{parts}

\part Find the two quotient groups for the series.  Find the ``standard'' abelian groups each is isomorphic to.

\begin{solution}
$\Z_{24}/\langle 12\rangle \cong \Z_{12}$ since the cosets are $\langle 12\rangle, \langle 12 \rangle + 1, \langle 12 \rangle + 2,\ldots \langle 12\rangle + 11$.  The other quotient group is $\langle 12 \rangle / \{0\} = \{0, 12\} \cong \Z_2$.
\end{solution}

\part For the quotient group that is not simple found above, find a non-trivial normal subgroup, and realize it as a quotient group $G'/\langle 12\rangle$ for some $G'$.  

\begin{solution}
We need to find a subgroup of $\Z_{12}$.  We could take $\{0, 4, 8\}$ for example.  In cosets, this corresponds to $\{\langle 12\rangle, \langle 12 \rangle + 4, \langle 12 \rangle + 8\}$ which is the result of taking $\langle 4 \rangle $ in $\Z_{24}$ and modding out by $\langle 12\rangle$.  Thus $\{0, 4, 8\} \cong \langle 4\rangle / \langle 12\rangle$.  

There are other correct solutions here: we could take $G'$ to be $\langle 6\rangle$, $\langle 3\rangle$ or $\langle 2 \rangle$ as well.
\end{solution}

\part Demonstrate/explain how this shows us how to build a longer normal series for $\Z_{24}$.

\begin{solution}
Using the quotient group we found in the previous part, we see that we can create a longer normal series:
\[\Z_{24} \supset \langle 4\rangle \supset \langle 12 \rangle \supset \{0\}\]
The $G'$ you find always allows you to find an intermediate normal subgroup since it will necessarily be a normal subgroup of $G$ and contain $H$.
\end{solution}



\end{parts}


\question[6] Find two different composition series for $\Z_{28}$.  Then use quotient groups to demonstrate that the two series are ``isomorphic'' (and explain what this means).

\begin{solution}
Here are some of the choices:
\[\Z_{28} \supset \langle 2\rangle \supset \langle 4 \rangle \supset \{0\}\]
\[\Z_{28} \supset \langle 7\rangle \supset \langle 14\rangle \supset \{0\}\]
\[\Z_{28} \supset \langle 2 \rangle \supset \langle 14 \rangle \supset \{0\}\]
(in fact, these are the only possibilities).

For the first one, the quotient groups are $\Z_2$, $\Z_2$ and $\Z_7$ (reading from left to right).  The second series has quotient groups $\Z_7$, $\Z_2$ and $\Z_2$.  The third: $\Z_2$, $\Z_7$, and $\Z_2$.  This is the way that the composition series are isomorphic: they have exactly the same quotient groups up to isomorphism and the order in which they occur.
\end{solution}



\question[4] Suppose $G$ is a group that contains a normal subgroup $H$ which is itself a non-abelian simple group.  Explain how you know that $G$ is not solvable.  Note, this is not difficult at all if you know the definitions of simple and solvable.

\begin{solution}
To say that $G$ is solvable means it has a composition series in which every quotient group is abelian.  To say $H$ is simple means it contains no non-trivial normal subgroups.  Now if $H$ is simple and is included in a composition series, then the composition series must end in $\ldots \supset H \supset \{e\}$.  The final quotient group will be $H/\{e\}\cong H$ which is not abelian.  By the Jordan-H\"older theorem, every composition series will be isomorphic to this one, so each will have a non-abelian quotient group.
\end{solution}


\question[6] Consider the polynomial $p(x) = x^7 - 1 = (x-1)(x^6+x^5+\cdots+x+1)$.  Let $\omega = e^{2\pi i/7}$ be a root of $p(x)$.  Then $\Q(\omega)$ is the splitting field for $p(x)$.
\begin{parts}
\part Explain how we know that the Galois group $\Gal(\Q(\omega):\Q)$ is isomorphic to $\Z_7^*$.  Give two examples of elements in $\Gal(\Q(\omega):\Q)$ and say what elements in $Z_7^*$ they correspond to.

\begin{solution}
Note that it makes sense to consider $\Gal(\Q(\omega):\Q)$ since $\Q(\omega)$ is the splitting field for $p(x)$.  Each element in the Galois group will be completely determined by where we send $\omega$ (since every other root is a power of $\omega$).  We can send $\omega$ to any of its 6 powers (including $\omega$, but not including $\omega^7 = 1$).  Thus there are 6 elements in the Galois group.  Further, if $\sigma(\omega) = \omega^k$ and $\tau(\omega) = \omega^j$ then
\[\sigma\tau(\omega) = \omega^{kj} = \omega^{jk} = \tau\sigma(\omega)\]
so the Galois group is abelian.  Thus we know that $\Gal(\Q(\omega):\Q) \cong \Z_6 \cong \Z_7^*$.  But considering $\Z_7^*$ is a little nicer since there we multiply number mod 7.  Here, we can think of each automorphism as multiplying the exponent by a number $\{1,2,\ldots, 6\}$ and since we simple travel around the 7 points on the unit circle, we do so mod 7.

Two specific elements might be $\sigma$ and $\tau$ where $\sigma(\omega) = \omega^2$ and $\tau(\omega) = \omega^3$.  These correspond to the elements 2 and 3 in $\Z_7^*$.

\end{solution}


  
\part $\Z_7^*$ has a composition series $\Z_7^* \supset \{1,6\} \supset \{1\}$.  Find the corresponding series of extension fields of $\Q$.  In other words, find the intermediate field $E$ such that $\Gal(\Q(\omega):E) \cong \{1,6\}$. 

\begin{solution}
Let $\beta \in \Gal(\Q(\omega):\Q)$ be the element corresponding to $6 \in \Z_7^*$.  Specifically $\beta(\omega) = \omega^6$.  Then $\beta$ is complex conjugation.  Now consider $\omega + \omega^6$.  This is not an element of $\Q$, but it is fixed by $\beta$.  So we can take $E = \Q(\omega + \omega^6)$.  Note that $\omega + \omega^6 = 2 \sin(3\pi/14)$ is a real number and a root of the polynomial $x^3+x^2 - 2x -1$ (thanks WolframAlpha!).  So $\Q(\omega + \omega^6)$ is a degree 3 extension of $\Q$.  Note that this means that $\Q(\omega)$ is a degree 2 extension of $\Q(\omega+\omega^6)$, which is not a surprise since $|\Gal(\Q(\omega):\Q(\omega + \omega^6))| = 2$.
\end{solution}


\end{parts}


\question[5] Find a degree 5 polynomial whose Galois group is isomorphic to $S_5$.  Explain how you know your example works.  Your example should be different from the one we discuss in class.

\begin{solution}
We must find an irreducible polynomial of degree 5 with exactly two non-real roots.  This will guarantee that the Galois group contains a 5-cycle (since the polynomial is irreducible, using Cauchy's theorem) and a 2-cycle (since complex conjugation switches just the two non-real roots), so contains all permutations in $S_5$.  

Such a polynomial is $p(x) = 3x^5 - 15x + 5$, which is irreducible by Eisenstein's criterion.  That it has exactly two non-real roots can be seen by graphing, or more carefully, by considering the derivative $15x^4 - 15$ which has exactly two real roots ($\pm 1$), so the original polynomial only has one maximum and one minimum.  Then use the intermediate value theorem to prove that there are roots between -2 and -1, between -1 and 1 and between 1 and 2 (specifically, $p(-2) = -61$ and $p(-1) = 17$ so there is a zero between $x = -2$ and $x = -1$; similarly for the other intervals).
\end{solution}


\bonusquestion[5000] Bonus: express the roots of the polynomial you found in the previous question in terms of rational numbers, field operations and roots (e.g., square roots, cube roots, etc.)
\begin{solution}
Since $S_5$ is not a solvable group, the polynomial will not be solvable by radicals. Thus it is impossible to complete this problem.  Hilarius, right?
\end{solution}

\end{questions}

\end{document}


