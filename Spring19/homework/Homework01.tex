\documentclass[11pt]{exam}

\usepackage{amsmath, amssymb, amsthm, multicol}
\usepackage{graphicx}
\usepackage{textcomp}
\usepackage{tikz}
\usepackage{mathrsfs}
%\usepackage[top=1in, bottom=1in, left=1in, right=1in]{geometry}
%\usepackage{framed}


%\newenvironment{ques}[2]{\vskip 1ex \noindent{\bf Question #1:}\marginpar{(#2 pts)}}{}
%\newenvironment{sol}{\begin{framed} \noindent{\em Solution:}}{\end{framed}\vskip 1em}



\def\d{\displaystyle}
\def\b{\mathbf}
\def\N{\mathbb{N}}
\def\R{\mathbb{R}}
\def\Z{\mathbb{Z}}
\def\Q{\mathbb{Q}}
\def\C{\mathbb{C}}
\def\F{\mathscr{F}}
\def\st{~:~}
\def\bar{\overline}
\def\inv{^{-1}}
\def\imp{\rightarrow}
\def\and{\wedge}
\def\onto{\twoheadrightarrow}

%\pointname{pts}
\pointsinmargin
\marginpointname{pts}
\marginbonuspointname{ bns}
\addpoints
\pagestyle{head}
%\printanswers

\firstpageheader{MATH 322}{\bf Homework 1}{Due: Wednesday, January 23}




\begin{document}
\noindent \textbf{Instructions}: Carefully write up solutions to the questions below.  A solution should consist of both the answer and a careful explanation for why that answer must be correct.  Any solutions without an explanation written out in English prose will receive no credit.
\vskip 1ex
You are encouraged to work with classmates, but your write up should be in your own words.  Also, other than our textbook, you may NOT use the internet to search for solutions.


\begin{questions}

\question[5] Find a single generator for the smallest ideal in $\Q[x]$ which contains the polynomials $x^3 + 3x^2 + 3x+2$ and $2x^3 - 3x^2 - 11x + 6$.  Explain how you know that this generator is in the ideal.

\begin{solution}
  Both polynomials are multiples of $x+2$, which we can find using the Euclidean Algorithm.  In fact, doing this gives $91/81 x + 182/81$, but this is a multiple of the simpler $x+2$.  We know that $x+2$ is in the ideal generated by the two polynomials because we can write it as a combination of them using Bezout's lemma.  Really, this is because in each stop of the Euclidean Algorithm, we have $a(x) = q(x) b(x) + r(x)$.  We have that $a(x)$ and $b(x)$ are in the ideal, so therefore $r(x) = a(x) -q(x)b(x)$ is as well.  
\end{solution}

\question[9] Consider the polynomial $p(x) = x^3 - 5$ in $\Q[x]$.  
\begin{parts}
\part Explain how we know that the quotient ring $\Q[x]/\langle x^3 - 5\rangle$ is actually a field.  That is, show that every non-zero element of the quotient ring has a multiplicative inverse.  Hint: you will want to use Bezout's lemma.
\begin{solution}
Let $a(x)$ be a polynomial in $\Q[x]$ which is not in $\langle x^3 - 5\rangle$, so that $\langle x^3 - 5\rangle+a(x) \ne \langle x^3 - 5\rangle$ (the zero element).  We will show that $\langle x^3 - 5\rangle+a(x)$ has a multiplicative inverse.  

Since $a(x) \notin \langle x^3 - 5\rangle$ we know that $a(x)$ is not a multiple of $x^3 - 5$.  Since $x^3 - 5$ is irreducible, this tells us that $a(x)$ and $x^3 - 5$ have gcd 1.  Thus by Bezout's lemma there are polynomials $s(x)$ and $t(x)$ such that 
\[(x^3 - 5)s(x) + a(x) t(x) = 1\]
But this says that \[1 \in \langle x^3 - 5\rangle + a(x)t(x) = (\langle x^3 - 5\rangle + a(x))(\langle x^3 - 5\rangle + t(x))\]
so $(\langle x^3 - 5\rangle + a(x))(\langle x^3 - 5\rangle+t(x)) = \langle x^3 - 5\rangle+1$.  But $\langle x^3 - 5\rangle+1$ is the multiplicative identity, so we have found an inverse for $\langle x^3 - 5\rangle+a(x)$.
\end{solution}


\part Let $E = \{a + b\sqrt[3]{5} + c\sqrt[3]{5}^2 \st a,b,c\in\Q\}$.  How does this set relate to the field $\Q[x]/\langle x^3 - 5\rangle$?  Be explicit (for example, if you say they are isomorphic, give the isomorphism).

\begin{solution}
Using our notation, we have the $E = \Q(\sqrt[3]{5})$ which we know is isomorphic to $\Q[x]/\langle x^3 - 5\rangle$ by the Fundamental Homomorphism Theorem.  The homomorphism from $\Q[x]$ onto $E$ is given by the evaluation map which sends a polynomial $a(x)$ to $a(\sqrt[3]{5})$.  The kernel of this homomorphism is the set of polynomials which have $\sqrt[3]{5}$ as a root.  In other words, all multiples of the minimum polynomial $x^3 - 5$.  

So what is the isomorphism from $E \to \Q[x]/\langle x^3 - 5\rangle$?  Well we need to send elements of $E$ to cosets.  Define: \[a + b\sqrt[3]{5} + c\sqrt[3]{5}^2 \quad \rightsquigarrow \quad \langle x^3 - 5\rangle + a + bx + cx^2\]
Notice that going backwards is exactly the evaluation map of ``plugging in $\sqrt[3]{5}$ into $a(x)$'' - the exact same map we defined for the homomorphism, only this time we are grouping all elements that are equivalent modulo $\langle x^3 - 5\rangle$.
\end{solution}

\part Find the element of $E$ (in $a + b\sqrt[3]{5} + c\sqrt[3]{5}^2$ form) equal to $1/(3-2\sqrt[3]{5} + \sqrt[3]{5}^2)$ using polynomials.  That is, use the relationship you described in part (b) so you can work in $\Q[x]/\langle x^3-5\rangle$ instead of in $E$.

\begin{solution}
We want to find the inverse of $\sqrt[3]{5}^2 - 2\sqrt[3]{5} + 3$, which under the isomorphism corresponds to $a(x) = x^2 - 2x + 3$.  Apply the Euclidean Algorithm to $a(x)$ and $x^3 - x$:
\[(x^3-5)=(x+2)(x^2-2x+3)+(x-11)\]

\[x^2-2x+3=(x+9)(x-11)+102.\] 

Solving backwards we get that 

\begin{align*}
102 = (x^2-2x+3)-(x+9)(x-11) & = (x^2-2x+3)-(x+9)((x^3-5)-(x+2)(x^2-2x+3)) \\ 
& = (1+(x+9)(x+2))(x^2-2x+3)-(x+9)(x^3-5)\\
& = (x^2+11x+19)(x^2-2x+3)-(x+9)(x^3-5)
\end{align*}

Now if we move back to $E$ (by plugging in $\sqrt[3]{5}$ in for $x$) we see that 
\[102 = (\sqrt[3]{5}^2+11\sqrt[3]{5}+19)(\sqrt[3]{5}^2 - 2\sqrt[3]{5} + 3)\]
so the inverse of $\sqrt[3]{5}^2 - 2\sqrt[3]{5} + 3$ is
\[\frac{1}{102}\sqrt[3]{5}^2+ \frac{11}{102}\sqrt[3]{5} + \frac{19}{102}\]
\end{solution}
\end{parts}


\question[6] Let $A$ be a commutative ring with unity.  Let $J$ be an ideal of $A$.  We say that $J$ is prime provided for any $a, b \in A$, if $ab \in J$ then $a \in J$ or $b \in J$.  
\begin{parts}
\part Prove that if $J$ is prime, then $A/J$ is an integral domain.
\begin{solution}
Suppose $J$ is prime.  Consider elements $J+a$ and $J+b$ in $A/J$, and suppose $(J+a)(J+b) = J$.  This says that $ab \in J$, and since $J$ is prime, we can conclude that either $a \in J$ or $b \in J$.  But this means either $J+a = J$ or $J+b = J$.  This proves that $A/J$ is an integral domain: given that two elements multiply to the zero element, either one or the other is zero, so there are no zero divisors.

\end{solution}

\part Prove that if $A/J$ is an integral domain, then $J$ is prime.
\begin{solution}
Suppose $A/J$ is an integral domain.  Let $ab \in J$ be given.  We have $(J+a)(J+b) = J+ab = J$.  But this means that $J+a = J$ or $J+b = J$ (since there are no zero divisors) so either $a \in J$ or $b \in J$.
\end{solution}
\end{parts}

\question[8] Let $A$ be a commutative ring with unity.  An ideal $J$ is \emph{proper} if $A \ne J$.  We say that a proper ideal $J$ is \emph{maximal} if no proper ideal of $A$ strictly contains $J$ (that is, if $K$ is a proper ideal of $A$ and $J \subseteq K$ then $J = K$).
\begin{parts}
\part Prove that if $J$ is maximal, then $A/J$ is a field (you may assume that $A/J$ is a commutative ring with unity).  Here are some hints: first, explain why you want to show that for any $a \notin J$, that there is some element $x$ such that $(J+a)(J+x) = J+1$.    Then let $K = \{xa+j \st x \in A, j \in J\}$, and prove that $K$ is an ideal strictly larger than $J$.  In particular, $1 \in K$.  Finally, explain why this is enough to finish the proof.

\begin{solution}
We want to show that every non-zero element of $A/J$ has an inverse.  So consider such an element $J+a$ (so $a \notin J$, since this should be non-zero).  Let $K = \{xa+j \st x \in A, j \in J\}$.  First, we claim $K$ is an ideal.  It is closed under subtraction: $x_1a+j_1 - (x_2a+j_2) = (x_1-x_2)a + (j_1 - j_2)$.  It also absorbs products: $(xa+j)\cdot b = xab + jb$.  This is in $K$ since $xb \in A$ and $jb \in J$ (as $J$ absorbs products).

Further, the ideal $K$ contains $J$, since for any $j \in J$, $j = 0a + j \in K$.  But $K \ne J$ since $1a + 0 = a \notin J$.  So $K$ is a strictly larger ideal than $J$, which implies that $K = A$, since $J$ is maximal.  In particular, $1 \in K$, so $1 = xa + j$ for some $x \in A$ and $j \in J$.  This means that $1 \in J + ax$ and thus $J+1 = J + ax = (J+a)(J+x)$.  We have found the inverse of $J+a$, it is $J+x$.

\end{solution}


\part Prove that if $A/J$ is a field, then $J$ is maximal. 
\begin{solution}
Assume $A/J$ is a field.  Suppose there is an ideal $K$ containing $J$ but strictly larger (we will show that $K = A$, which shows that $J$ is maximal).  So there is some element $a \in K$ not in $J$.  In particular, $J+a \ne J$, so this coset has a multiplicative inverse, call it $J+a'$.  We have $(J+a)(J+a') = J+aa' = J+1$.  This tells us that $1 \in J+aa'$ and as such $1 = j+aa'$.  But $j \in K$ and $a \in K$ so $1 = j+aa' \in K$.  The only ideal that contains 1 is the trivial ideal $A$, so we are done.
\end{solution}

\end{parts}


\question[2] Assuming the results from the previous two questions, prove that every maximal ideal is prime.  This should be a 3-sentence proof.

\begin{solution}
Let $J$ be a maximal ideal.  Then $A/J$ is a field, and since every field is an integral domain, $A/J$ is also an integral domain.  But then we have that $J$ must be prime.
\end{solution}

\bonusquestion[3] Bonus: it is not true that every prime ideal is maximal (although this does hold for $\Z$ and for $F[x]$).  Find an example of a ring $A$ with an ideal $J$ that is prime but not maximal.  Justify your answer.  Hint: look at a polymoial ring that for which the coefficients do not belong to a field.
\begin{solution}
Consider $\Z[x]/\langle x\rangle$.  We can argue that this quotient ring is isomorphic to $\Z$, so we have that $\langle x\rangle$ is not maximal ($\Z$ is not a field).  But $\langle x\rangle$ is prime (do you see why?).
\end{solution}
\end{questions}

\end{document}


