\documentclass[11pt]{exam}

\usepackage{amsmath, amssymb, amsthm, multicol}
\usepackage{graphicx}
\usepackage{textcomp}
\usepackage{tikz}
\usepackage{mathrsfs}
%\usepackage[top=1in, bottom=1in, left=1in, right=1in]{geometry}
%\usepackage{framed}


%\newenvironment{ques}[2]{\vskip 1ex \noindent{\bf Question #1:}\marginpar{(#2 pts)}}{}
%\newenvironment{sol}{\begin{framed} \noindent{\em Solution:}}{\end{framed}\vskip 1em}



\def\d{\displaystyle}
\def\b{\mathbf}
\def\N{\mathbb{N}}
\def\R{\mathbb{R}}
\def\Z{\mathbb{Z}}
\def\Q{\mathbb{Q}}
\def\C{\mathbb{C}}
\def\F{\mathscr{F}}
\def\st{~:~}
\def\bar{\overline}
\def\inv{^{-1}}
\def\imp{\rightarrow}
\def\and{\wedge}
\def\onto{\twoheadrightarrow}
\DeclareMathOperator{\Gal}{Gal}
\DeclareMathOperator{\ord}{ord}
\DeclareMathOperator{\lcm}{lcm}


%\pointname{pts}
\pointsinmargin
\marginpointname{pts}
\marginbonuspointname{bns-pts}
\addpoints
\pagestyle{head}
\printanswers

\firstpageheader{MATH 322}{\bf Homework 5 Solutions}{Due: Wednesday, March 20}




\begin{document}

\noindent \textbf{Instructions}: Same rules as usual.  Work together, write-up alone, no internet!
\vskip 1ex

\begin{questions}
  
  \question[6] Consider the 362880 elements in $S_9$.
  \begin{parts}
  \part What are the possible orders of elements in $S_9$.  For each possible order, give an example of an element with that order.  Explain how you know you have them all.
  \begin{solution}
  We know that the order of a cycle is its length, so there are elements of all orders in $\{1,2,\ldots, 9\}$, namely $(1)$, $(12)$, $(123)$, \ldots, $(123456789)$.  We can also get orders that are least common multiples of the lengths of disjoint cycles.  This way we can get $6$ (again), 10 as $(12)(34567)$, 14 as $(12)(3456789)$, 12 as $(123)(4567)$, 15 as $(123)(45678)$ and 20 as $(1234)(56789)$.  But that is all. 

  \end{solution}
  \part Give an example of an element with order 3 that does not fix any element of $\{1,2,\ldots,9\}$.
  \begin{solution}
  $(123)(456)(789)$ leaves no element fixed, but has order 3.
  \end{solution}
  \part What do elements of order 8 look like?  Bonus: how many elements of order 8 are there?
  \begin{solution}
  The only order 8 elements are the 8-cycles.  This is because any set of numbers with least common multiple 8 must include 8 itself.  

  To count these, not that since we always start a cycle with the smallest number in it, the cycle must start with either a 1 or a 2.  If it starts with a 1, there are $8!$ choices for the rest of the cycle (8 choices for the next number, 7 for the one after that, and so on until we have picked 7 numbers, so this is really $P(8,7) = 8!/1!$).  If it starts with a 2, then there cannot be a 1 in the cycle, so there are $7!$ ways to finish the cycle.  Thus there are $8! + 7! = 45360$ elements with order 8 (which happens to be 1/8 of all cycles in $S_9$).
  \end{solution}
  \end{parts}

  \question[8] Prove the following basic facts about orders of elements.  None of these are particularly difficult, so you should put most of your effort into writing a nice, clean proof of the fact.  In each of the following, $a$ is an element of a group $G$.
  \begin{parts}
  \part If $\ord(a) = n$ then for any $r < n$, $a^{n-r} = (a^r)\inv$.
  \begin{solution}
  \begin{proof}
  Let $a$ be an element of a group with $\ord(a) = n$, and let $r < n$.  Now $a^{n-r}\cdots a^r = a^{n-r+r} = a^n = e$ so $(a^r)\inv = a^{n-r}$.
  \end{proof}
  \end{solution}

  \part The order of $a\inv$ is the same as the order of $a$.
  \begin{solution}
  \begin{proof}
  Again let $\ord(a) = n$.  Now $(a\inv)^n = (a^n)\inv = e\inv = e$ so we know that $\ord(a\inv)$ is at most $n$.  We must also argue that no smaller $k$ has $(a\inv)^k = e$.  But if there were such a $k < n$, then we would have $(a^k)\inv = e$.  The only element that has $e$ as an inverse is $e$ itself, so this would say $a^k = e$, a contradiction since $n$ was the least positive such exponent.
  \end{proof}
  \end{solution}
  \part If $a^k = e$ where $k$ is odd, then the order of $a$ is odd.
  \begin{solution}
  \begin{proof}
  Suppose $a^k = e$ for some odd number $k$.  This does not mean that $k$ is the order of $a$, but we do know that $k$ will be a multiple of the order of $a$.  If $\ord(a)$ were even, then every multiple of that order would be even as well.  Since $k$ is an odd multiple of the order, we know the order must be odd as well.
  \end{proof}
  \end{solution}
  \part If $a\ne e$ and $a^p = e$ where $p$ is prime, then $\ord(a) = p$.
  \begin{solution}
  \begin{proof}
  Suppose $a \ne e$ and $a^p = e$ where $p$ is prime.  This does not mean that $\ord(a) = p$ right away, just that $p$ is a multiple of $\ord(a)$.  But since $p$ is prime, $p$ is only a multiple of $1$ and $p$.  We know that $\ord(a) \ne 1$ since $a \ne e$.  Thus $\ord(a) = p$.
  \end{proof}
  \end{solution}
  \end{parts}


  \question[12] Let $a$ and $b$ be elements of a group $G$ with $\ord(a) = m$ and $\ord(b) = n$.
  \begin{parts}
  \part Assume $a$ and $b$ commute.  Let $k = \ord(ab)$ and $p = \lcm(m,n)$.  Prove $k$ divides $p$.
  \begin{solution}
  \begin{proof}
  Assume $\ord(ab) = k$.  Consider $(ab)^p = a^pb^p$ (since $a$ and $b$ commute).  But $p$ is a multiple of $m$ and of $n$ so this means $(ab)^p = a^pb^p = e\cdot e = e$.  But this means that $p$ is a multiple of $\ord(ab)$, as needed.

  Note you could also prove this from scratch using the division algorithm (which is how we know that the only exponents that give the identity are multiples of the order).
  \end{proof}
  \end{solution}

  \part Assume $m$ and $n$ are relatively prime (i.e., $\gcd(m,n) = 1$).  Prove that no power of $a$ is equal to any power of $b$ (other than $e$).
  \begin{solution}
  \begin{proof}
  Suppose that $a^k = b^j$.  But then $(a^k)^n = (b^j)^n = (b^n)^j = e^j = e$, and similarly $(b^j)^m = (a^k)^m = (a^m)^k = e^k = e$.  This proves that $m$ and $n$ are both multiples of the order of $a^k$ (which is the same as the order of $b^j$).  But since the only number that $m$ and $n$ are both multiples of is 1, we have that $a^k = e = b^j$.  That is, if any power of $a$ is equal to a power of $b$, then those powers are the identity.  
  \end{proof}
  \end{solution}

  \part Use the previous parts to prove that if $a$ and $b$ commute and $m$ and $n$ are relatively prime, then $\ord(ab) = mn$.
  \begin{solution}
  \begin{proof}
  From the previous parts we know that $\ord(ab)$ divides $\lcm(m,n) = mn$, and that no power of $a$ is equal to a power of $b$ (other than $e$).  Again let $k = \ord(ab)$.  We have $(ab)^k = a^kb^k = e$, or in other words $a^k = b^{-k}$.  By part (b), this implies that $a^k = e = b^{-k} = b^k$.  But then $k$ is a multiple of both $m$ and $n$, so is also a multiple of $\lcm(m,n)$.  The only multiple of $\lcm(m,n)$ that is also a divisor of $\lcm(m,n)$ is $\lcm(m,n)$ itself.  Thus $k = mn$.
  \end{proof}
  \end{solution}

  \part Give an example to show that part (a) is not true if $a$ and $b$ do not commute.
  \begin{solution}
  For example, $a= (12)$ and $b = (13)$.  Then $ab = (132)$ and $\ord(ab) = 3$.  However, $\lcm(2,2) = 2$ and $3$ does not divide $2$.
  \end{solution}
  \end{parts}




  
  \question[4] Suppose $G$ is a group and $H$ and $K$ are distinct subgroups both with order the same prime number $p$.  Prove that $H \cap K = \{e\}$.  Hint: use Lagrange's theorem.

  \begin{solution}
  Consider an element $a \in H \cap K$.  Since $H$ is a group of order $p$, every element in $H$ (including $a$) must have order dividing $p$, by Lagrange's theorem.  Since $p$ is prime, this means $\ord(a) = p$ or $\ord(a) = 1$.  If $\ord(a) = p$, then the $p$ different powers of $a$ all belong to $H$, and also to $K$.  But since both $H$ and $K$ only have $p$ different elements, this means that $\langle a \rangle = H = K$ and the subgroups are not distinct.  Thus we have that $\ord(a) = 1$, so $a = e$.  In other words, the only element in both $H$ and $K$ is the identity.

  An alternative proof would be to recall that $H \cap K$ is a subgroup of $G$, and since $H \cap K \subseteq H$ also a subgroup of $H$ (and similarly of $K$).  Since $H$ has order $p$, we know by Lagrange's theorem that the order of $H\cap K$ is either $p$ or 1.  If it is $p$, then $H \cap K = H = K$, contradicting the assumption that $H$ and $K$ are distinct.  Thus $|H\cap K| = 1$ so $H \cap K = \{e\}$.
  \end{solution}


\end{questions}

\end{document}


