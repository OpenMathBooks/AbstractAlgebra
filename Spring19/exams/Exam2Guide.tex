\documentclass[10pt]{exam}

\usepackage{amsmath, amssymb, amsthm, mathrsfs, multicol, wasysym}
\usepackage{graphicx}
\usepackage{textcomp}
\usepackage{tikz}
\usepackage{answers}

\renewcommand{\labelitemi}{\Large\Square}
\renewcommand{\labelitemii}{\Large\Circle}
\def\d{\displaystyle}
\def\uplevel#1{\end{itemize}#1\begin{itemize}}
\def\deg{^\circ}
\def\st{~:~}

\def\imp{\rightarrow}
\def\Imp{\Rightarrow}
\def\iff{\Leftrightarrow}
\def\Iff{\Leftrightarrow}
\def\land{\wedge}



\def\N{\mathbb N}
\def\Z{\mathbb Z}
\def\R{\mathbb R}
\def\Q{\mathbb Q}
\def\P{\mathbb P}
\def\E{\mathbb E}
\def\O{\mathbb O}
\def\F{\mathbb F}
\def\C{\mathbb C}
\def\F{\mathscr{F}}
\def\Rp{\R^{\mathrm{pos}}}
\def\FR{\mathscr{F}(\R)}
\def\CR{\mathscr{C}(\R)}
\def\DR{\mathscr{D}(\R)}
\def\pow{\mathcal{P}}
\def\b{\mathbf}
\def\Im{\mathrm{Im}}
\def\st{~:~}
\def\bar{\overline}
\def\inv{^{-1}}
\DeclareMathOperator{\wgt}{wgt}
\DeclareMathOperator{\ord}{ord}
\DeclareMathOperator{\cis}{cis}
\DeclareMathOperator{\Gal}{Gal}
\DeclareMathOperator{\lcm}{lcm}





%\pointname{pts}
\pointsinmargin
\marginpointname{pts}
\addpoints
\pagestyle{head}
%\printanswers


\def\filename{Exam2Guide}
\def\doctitle{Exam 2 Study Guide}
\def\docdate{Spring 2019}

\def\ansfilename{\filename-solutions}


				
\Opensolutionfile{\ansfilename}
\Newassociation{answer}{Ans}{\ansfilename}
\Writetofile{\ansfilename}{\protect\documentclass[10pt]{exam} }
\Writetofile{\ansfilename}{\protect\usepackage{answers, amsthm, amsmath, amssymb, mathrsfs}
  \protect\pagestyle{head}
  \protect\firstpageheader{Math 322}{\bf \doctitle \\ Hints and Answers}{\docdate}
  \protect\Newassociation{answer}{Ans}{\ansfilename}
  }
  
\begin{Filesave}{\ansfilename}
\def\d{\displaystyle}
\def\uplevel#1{\end{itemize}#1\begin{itemize}}
\def\deg{^\circ}
\def\st{~:~}

\def\imp{\rightarrow}
\def\Imp{\Rightarrow}
\def\iff{\Leftrightarrow}
\def\Iff{\Leftrightarrow}
\def\land{\wedge}



\def\N{\mathbb N}
\def\Z{\mathbb Z}
\def\R{\mathbb R}
\def\Q{\mathbb Q}
\def\P{\mathbb P}
\def\E{\mathbb E}
\def\O{\mathbb O}
\def\F{\mathbb F}
\def\C{\mathbb C}
\def\F{\mathscr{F}}
\def\Rp{\R^{\mathrm{pos}}}
\def\FR{\mathscr{F}(\R)}
\def\CR{\mathscr{C}(\R)}
\def\DR{\mathscr{D}(\R)}
\def\pow{\mathcal{P}}
\def\b{\mathbf}
\def\Im{\mathrm{Im}}
\def\st{~:~}
\def\bar{\overline}
\def\inv{^{-1}}
\DeclareMathOperator{\wgt}{wgt}
\DeclareMathOperator{\ord}{ord}
\DeclareMathOperator{\cis}{cis}
\DeclareMathOperator{\Gal}{Gal}
\DeclareMathOperator{\lcm}{lcm}


\usepackage{tikz, multicol}
\renewenvironment{Ans}[1]{\setcounter{question}{#1}\addtocounter{question}{-1}\question }{}
\begin{document}
 \begin{questions}
\end{Filesave}


\firstpageheader{Math 322}{\bf \doctitle}{\docdate}


\begin{document}
The second exam  will cover all the material we have discussed since the first exam.  This means basically groups, although recall we have also discussed how groups relate to fields and polynomials.  Note that while field extensions was covered on the previous exam, you should be familiar enough with that material to answer questions about the relationship between groups and field (Galois Theory).  So in that sense, this is really a cumulative exam.  Here is a checklist of these topics.  

\begin{itemize}
  \item Cayley's Theorem (groups isomorphic to subgroups of $S_n$).
  \item Working in $S_n$
  \begin{itemize}
	  \item Cycle notation.
	  \item Products, powers, inverses, etc.
	  \item Writing permutations as products of disjoint cycles.
	  \item Even and odd permutations; transpositions.
	  \item Orders of permutations
  \end{itemize}
  \item The order of elements (don't forget the division algorithm).
  \item Cyclic groups and the relationship to order.
  \item The theorems of Lagrange and Cauchy.
	\item Fermat's Little and Euler's Theorem, and their relationship to RSA cryptography.
  \item $p$-groups.
  \item The Fundamental Theorem of Finite Abelian Groups.
  \item Subnormal series, composition series, solvable groups.
  \item Galois correspondence.
  \item Solvability by radicals

\end{itemize}


The homework and class activities should give you a good idea of the types of questions to expect.  Also take a look at the quizzes we did over this material.  Copies of these assignments, with solutions, are available on Canvas.

Additionally, the questions below would all make fine exam questions.\footnote{Disclaimer: Question on the actual exam may be easier or harder than those given her.  There might be types of questions on this study guide not covered on the exam and questions on the exam not covered in this study guide.  Questions on the exam might be asked in a different way than here.  If solving a question lasts longer than four hours, contact your professor immediately.}  



\subsection*{Sample Questions}


\begin{questions}


\question The group $D_3$ of symmetries of the triangle is isomorphic to $S_3$.  But by Cayley's theorem, the group is also isomorphic to a {\em subgroup} of $S_6$.  Find such a subgroup (using the proof of Cayley's theorem).

\begin{answer}
  Here is the table for $D_3$:
  
          \def\a{$R_{0}$}
      \def\b{$R_{120}$}
      \def\c{$R_{240}$}
      \def\d{$F_v$}
      \def\e{$F_x$}
      \def\f{$F_{-x}$}
   
 \begin{tabular}{c|cccccc}
  $\ast$ & \a & \b & \c & \d & \e & \f \\ \hline
  \a & \a & \b & \c & \d & \e & \f \\
  \b & \b & \c & \a & \e & \f & \d \\
  \c & \c & \a & \b & \f & \d & \e \\
  \d & \d & \f & \e & \a & \c & \b \\
  \e & \e & \d & \f & \b & \a & \c \\
  \f & \f & \e & \d & \c & \b & \a
 \end{tabular}
 
 For each of the elements of $D_3$, we find a permutation in $S_6$ (6 because there are 6 elements of $D_3$). The identity permutation is $\pi_{R_0} = \begin{pmatrix}1 & 2 & 3 & 4 & 5 & 6 \\ 1 & 2 & 3 & 4 & 5 & 6\end{pmatrix}$.   Now to find $\pi_{R_{120}}$ we look at the result of multiplying each element by $R_{120}$ (on the left).  This gives
 \[\pi_{R_{120}} = \begin{pmatrix} 1 & 2 & 3 & 4 & 5 & 6 \\ 2 & 3 & 1 & 5 & 6 & 4\end{pmatrix}\]
 This is because if we multiply the first element of $D_3$ ($R_0$) by $R_{120}$ we get $R_{120}$ (the second element of $D_3$).  If we multiply element 2 by $R_{120}$ we get element 3 (since $R_{120}R_{120} = R_{240}$).  If we multiply element 3 by $R_{120}$ we get element 1.  If we multiply element 4 ($F_{v}$) by $R_{120}$ we get element 5 ($F_x$).  And so on.
 
 Here are the other permutations:
 \[\pi_{R_{270}} = \begin{pmatrix} 1 & 2 & 3 & 4 & 5 & 6 \\ 3 & 1 & 2 & 6 & 4 & 5\end{pmatrix} \qquad \pi_{F_v} = \begin{pmatrix}1 & 2 & 3 & 4 & 5 & 6 \\ 4 & 6 & 5 & 1 & 3 & 2\end{pmatrix}\]
 \[\pi_{F_{x}} = \begin{pmatrix} 1 & 2 & 3 & 4 & 5 & 6 \\ 5 & 4 & 6 & 2 & 1 & 3 \end{pmatrix} \qquad \pi_{F_{-x}} = \begin{pmatrix}1 & 2 & 3 & 4 & 5 & 6 \\ 7 & 6 & 4 & 3 & 2 & 1\end{pmatrix}\]
\end{answer}



\question The identity can be written as $\varepsilon = (13)(24)(35)(14)(12)(15)(34)(45)$.
Mimic the proof that $\varepsilon$ must be even and show how to eliminate $x = 5$ from the product of transpositions and write $\varepsilon$ as the product of 2 fewer transpositions in the process.  Show all intermediate steps.

\begin{answer}
  We look for the last occurrence of 5, which is in the final transposition.  Now $(34)(45) = (35)(34)$, so we can write
  \[\varepsilon = (13)(24)(35)(14)(12)(15)(35)(34)\]
  Now $(15)(35) = (35)(13)$ so we gets
   \[\varepsilon = (13)(24)(35)(14)(12)(35)(13)(34)\]
   Then $(12)(35) = (35)(12)$ since the transpositions are disjoint so,
   \[\varepsilon = (13)(24)(35)(14)(35)(12)(13)(34)\]
   and similarly
   \[\varepsilon = (13)(24)(35)(35)(14)(12)(13)(34)\]
   But $(35)(35) = (1)$ so we end up with
   \[\varepsilon = (13)(24)(14)(12)(13)(34)\]
\end{answer}





\question Consider the following permutations in $S_5$
\[f = \begin{pmatrix}1 & 2 & 3 &4 &5 \\ 3 & 2 & 4 & 1 & 5\end{pmatrix} \qquad g = \begin{pmatrix}1 & 2 & 3 &4 &5 \\ 2 & 4 & 3 & 5 & 1\end{pmatrix}\]

\begin{parts}
  \part Find $fg$.
  \part Find $f\inv$
  \part Find $f g f\inv$
  \part Find $f^3 g^2 f g\inv$
  \part Write $f$ and $g$ as cycles or the product of disjoint cycles.
  \part Write $f$ and $g$ as products transpositions
  \part Write all the answers to parts (a)-(d) both as cycles or the product of disjoint cycles and as the product of transpositions
\end{parts}

\begin{answer}
 \begin{parts}
  \part  $fg = \begin{pmatrix} 1 & 2 & 3 & 4 & 5 \\ 2 & 1 & 4 & 5 & 3 \end{pmatrix}$.
  \part $f\inv = \begin{pmatrix} 1 & 2 & 3 & 4 & 5 \\ 4 & 2 & 1 & 3 & 5 \end{pmatrix}$
  \part $f g f\inv = \begin{pmatrix} 1 & 2 & 3 & 4 & 5 \\ 5 & 1 & 2 & 4 & 3 \end{pmatrix}$
  \part $f^3 g^2 f g\inv = \begin{pmatrix} 1 & 2 & 3 & 4 & 5 \\ 2 & 3 & 1 & 5 & 4 \end{pmatrix}$
  \part $f = (134)$,  $g = (1245)$.
  \part $f = (13)(34)$, $g = (12)(24)(45)$ 
  \part $fg = (12)(345) = (12)(34)(45)$, $f\inv = (143) = (14)(34)$, $fgf\inv = (1532) = (15)(53)(32)$, $f^3g^2fg\inv = (123)(45) = (12)(23)(45)$.
\end{parts}
\end{answer}








\question Consider the group $S_5$, which has 120 elements.  
\begin{parts}
    \part If $\alpha \in S_5$, find $\alpha^{120}$. Explain.
    \part Is there an element of order $120$?  Explain.
    \part Is there an element of order 6?  Explain.
    \part Is there an element of order 7?  Explain.


\end{parts}


  \begin{answer}
  \begin{parts}
    \part By Lagrange's theorem, the order of $\alpha$ must divide 120, the order of $S_5$.  Thus $120$ is a multiple of the order of $\alpha$. But $\alpha^t = (1)$ if and only if $t$ is a multiple of $\ord(\alpha)$, so $\alpha^{120} = (1)$.
    \part No, since $S_5$ is not cyclic.  If there were an element of 120, then that element would generate $S_5$.
    \part Yes.  For example, $(12)(345)$ has order 6.
    \part No.  By Lagrange's theorem, the order of any element must divide the order of the group.  120 is not divisible by 7.
  \end{parts}
  \end{answer}




\question Let $G$ be a group with elements $a$ and $b$.
\begin{parts}
	\part True or false: $\ord(ab) = \ord(a)\ord(b)$?  Explain why or give a counterexample.
	\part True or false: $\ord(ab) = \lcm(\ord(a),\ord(b))$?  Explain why or give a counterexample.
	\part True or false: $\ord(ab) | \ord(a)\ord(b)$?  Explain why or give a counterexample.
	\part For any of the three statements above which are false, are they true if $a$ and $b$ commute?  What else would you need to assume to make the statements true?
\end{parts}

	\begin{answer}
		\begin{parts}
		\part False. for example, if $a$ and $b$ are inverses of each other, then $\ord(ab) = 1$ but if $a \ne e$ then $\ord(a)\cdot \ord(b) > 1$.
		\part False as well.  The same counterexample works.
		\part This is also false.  For example, $a = (12)$ and $b = (13)$ in $S_3$.  Then $\ord(ab) = 3$ but $\ord(a)\ord(b) = 4$.		
		
		\part For any of these to be true, we need $a$ and $b$ to commute.  In this case, part (c) is true, but the other two are still false (since an element and its inverse commute).  To make (a) and (b) true, we could  insist the orders be relatively prime.  
		\end{parts}
	\end{answer}





\question Let $G$ be a cyclic group generated by $a$ and let $H$ be a subgroup of $G$.  Prove that $G/H$ is cyclic. 

	\begin{answer}
		If $H = G$ we are done (since $G/H$ would be trivial).  So assume $a \notin H$.  We claim that $Ha$ generates $G/H$.  Since every element in $G/H$ is $Hb$ for some element $b \in G$, and every element in $G$ is a power of $a$, we have every element of $G/H$ is $H(a^k)$ for some $k$.  But $H(a^k) = (Ha)^k$ so every element of $G/H$ is a power of $Ha$.  In other words, $G/H$ is cyclic.
	\end{answer}





\question Suppose $G = \langle a \rangle$ is cyclic with order $n$.  
\begin{parts}
  \part What must be true of any subgroup $H$ of $G$?
  \part Prove that for every $k$ which divides $n$, there is a subgroup of order $k$.
  \part Is part (b) necessarily true if $G$ is not cyclic?  Hint: look at the subgroups of $A_4$.
\end{parts}


  \begin{answer}
  \begin{parts}
    \part $H$ must be cyclic, and must have order dividing $n$.
    \part Let $H = \langle a^{n/k}\rangle$.  Then $H$ has order $k$, since its generator has order $k$.
    \part No.  For example, $A_4$ has order 12 but has no subgroup of order 6.
  \end{parts}

  \end{answer}





\question Consider the group $\Z_{17}$.  Find all subgroups and justify your answer.

  \begin{answer}
   The subgroups are just $\{0\}$ and $\Z_{17}$.  Since the order of $\Z_{17}$ is 17, a prime number, the order of any subgroup can only be 1 or 17.  
  \end{answer}

  

%\question Consider the cyclic group $\Z_{20}$.
%\begin{parts}
%  \part What is the order of 16?
%  \part Find the subgroup $\langle 16 \rangle$.
%  \part List all the cosets of $\langle 16 \rangle$.  What is its index in $\Z_{20}$?
%  \part Is $\langle 16 \rangle$ a normal subgroup of $\Z_{20}$?  Explain.
%  \part Find a homomorphism from from $\Z_{20}$ to some group $H$ which has $\langle 16 \rangle$ as its kernel.  What is $H$ isomorphic to?
%\end{parts}
%
%
%\begin{answer}
% \begin{parts}
%   \part $\ord(16) = 5$.
%   \part $\langle 16 \rangle = \{0, 4, 8, 12, 16\}$
%   \part The index is 4.  The cosets are $\langle 16 \rangle$, $\langle 16 \rangle + 1 = \{1, 5, 9, 13, 17\}$, $\langle 16\rangle + 2 = \{2, 6, 10, 14, 18\}$, and $\langle 16 \rangle + 3 = \{3, 7, 11, 15, 19\}$.
%   \part Yes it is.  In fact, every subgroup of $\Z_{20}$ is normal, because $\Z_{20}$ is abelian.
%   \part Define $f:\Z_{20} \to \Z_{4}$ by $f(n) = r$ where $r$ is the remainder when $n$ is divided by 4.  So $f(0) = 0$, $f(1) = 1$, $f(2) = 2$, $f(3) = 3$, $f(4) = 0$, $f(5) = 1$, etc.
% \end{parts}
%
%\end{answer}



\question State Lagrange's theorem and explain what it means.  Briefly explain how we know it is true (do not give a full proof -- just explain the key steps).  

\begin{answer}
 If $G$ is a finite group and $H$ is a subgroup of $G$, then the order of $H$ is a factor of the order of $G$.  We know this is true because we can always form the cosets of $H$, which form a partition of $G$. Each coset has the same number of elements as $H$.  Thus the number of cosets (i.e., the index of $H$ in $G$) is $|G|/|H|$.  Since this is a whole number, $|H|$ must divide evenly into $|G|$. 
\end{answer}


\question State Cauchy's theorem and explain what it means.  Briefly explain how we proved Cauchy's theorem for abelian groups (do not give the full proof -- just explain the key steps).

\begin{answer}
Cauchy's theorem is a sort of converse of Lagrange's theorem (the converse of Lagrange's theorem would be that if $n$ divides the size of $G$, then there is a subgroup $H$ of $G$ with order $n$, but this is false, so Cauchy's theorem is the best we can hope for).  Specifically, if $p$ is a prime that divides the order of a finite group $G$, then there is an element of $G$ which has order $p$.  We proved this in the abelian case by induction on the size of the group.  For the inductive case, if there wasn't an element of order $p$, then we could take any element whose order was not an element of $p$ and consider the cyclic group generated by this element.  We then modded out by this subgroup to get a quotient group which was of smaller size, but still a size divisible by $p$.  Thus the quotient group would have an element of order $p$ (by the inductive hypothesis) and we could use that to move back to the original group and find an element of order $p$ (by first finding an element whose order was a multiple of $p$).
\end{answer}




\question Find all abelian groups of order 480.

\begin{answer}
Note that $480 = 2^5 * 3 * 5$.  By the Fundamental Theorem of Finite Abelian Groups, we know that we can write any abelian group as the direct product of cyclic $p$-groups.  In this case, those $p$-groups will be 2-groups, 3-groups and 5-groups.  The 3-groups and 5-groups must be $\Z_3$ and $\Z_5$ respectively.  The $2$-groups can be one of
\[\Z_{2^5} \qquad \Z_{2^4}\times \Z_2 \qquad \Z_{2^3}\times \Z_{2^2} \qquad \Z_{2^3} \times \Z_{2}\times\Z_2\]
\[\Z_{2^2}\times \Z_{2^2}\times \Z_2 \qquad \Z_{2^2} \times\Z_2\times \Z_2\times\Z_2 \qquad \Z_2\times\Z_2\times\Z_2\times\Z_2\times\Z_2\]
Which case we are in depends on the distribution of elements of particular order.  For example, if there are elements of order 32, we will be looking at $\Z_{2^5}\times\Z_3\times\Z_5$.  If there are no elements of order 4, we will be in
\[\Z_2\times\Z_2\times\Z_2\times \Z_2\times\Z_2\times \Z_3\times\Z_5.\]
\end{answer}






\question Suppose the group $G$ has subnormal series 
\[G \supset H \supset \{e\}\]
and that $G/H \cong \Z_{10}$.  Assume also that $H$ is simple.
\begin{parts}
\part Explain how we know that the above series is not a composition series.
\part Explain how we could find two different composition series for $G$.
\part Prove that if $H$ is abelian, then $G$ is solvable.
\part If $G$ happens to be the Galois group for some field $E$ over $\Q$, what can you say about subfields of $E$?
\end{parts}

	\begin{answer}
		\begin{parts}
		\part The series is not a composition series (by definition) since not all of the quotient groups are simple.  $G/H \cong \Z_{10}$ has non-trivial normal subgroups.
		\part We can find a non-trivial subgroup of $\Z_{10}$ and use that to find a subgroup of $G$ containing $H$.  There are two non-trivial subgroups of $\Z_{10}$, $\langle 2\rangle$ and $\langle 5\rangle$, and each of these will produce a different intermediate group.  Specifically, let $K_1$ be a subgroup of $G$ such that $K_1/H \cong \Z_5 \cong \langle 2\rangle$ (as a subgroup of $\Z_{10}$) and $K_2$ be a subgroup of $G$ such that $K_2/H \cong \Z_2 \cong \langle 5\rangle$ (as a subgroup of $\Z_{10}$).  Note that $G/K_1 \cong \Z_2$ and $G/K_2 \cong \Z_{5}$.  We get the two composition series
		\[G \supset K_1 \supset H \supset \{e\}\]
		\[G \supset K_2 \supset H \supset \{e\}\]
		These are composition series since all the quotient groups are simple (we know $H/\{e\}$ is simple because $H$ is).
		\part Since $H \cong H/\{e\}$, we have that all the quotient groups of the two series above are abelian.  Since the series above are composition series, all composition series are isomorphic to these, so all of their quotient groups are also abelian.  This is the definition of a group being solvable.
		\part $E$ will have subfields corresponding to the subgroups of $G$: the fixed fields of the subgroups.  Call these $F_{K_1}$, $F_{K_2}$ and $F_H$.  We match them up by what elements are fixed by the automorphisms in the subgroup.  So the subgroup $K_1$ contains automorphisms of $E$ that fix $F_{K_1}$, for example.  Viewed this way, $K_1 \cong \Gal(E:F_{K_1})$.  Note that in using this notation, $F_{G} = \Q$ and $F_{\{e\}} = E$. We get two sequences of fields: $\Q \subset F_{K_1} \subset F_H \subset E$ and $\Q \subset F_{K_2} \subset F_{H} \subset E$.  Note in particular that $F_{K_1}$ is smaller than $F_H$, precisely because $K_1$ is \emph{larger} than $H$.  Since $G/K_1 \cong \Z_2$ we see that $F_{K_1}$ is a degree 2 extension of $\Q$.  We also know that $F_H$ will be a degree $5$ extension of $F_{K_1}$.  We can do a similar analysis for the other chain of fields.  Since we do not know what the size of $H$ is, we do not know what the degree of $E$ over $F_{H}$ is (although since $H$ is simple and abelian, we can be sure that it is a prime number).		
		\end{parts}
	\end{answer}


\question Consider the polynomial $p(x) = x^3 + 5x^2 - 10x + 15$.  Let $E$ be the splitting field for $p(x)$ and $G$ be the Galois group of $E$ over $\Q$.
\begin{parts}
	\part Prove that $G$ contains an element of order 3.
	\part Prove that $G$ contains an element of order 2.
	\part Explain how we know that there is a intermediate field $I$ strictly between $\Q$ and $E$ that is the splitting field for a polynomial.  What can you say about this field?
	\part Explain how you know that $G \cong S_3$ and not to $\Z_6$.
	\part Does the argument above prove that $p(x)$ is not solvable by radicals?  Is $p(x)$ solvable by radicals?
\end{parts}

\begin{answer}
	\begin{parts}
		\part Since $p(x)$ is irreducible in $\Q$, we know that $\Q(\alpha)$ will have degree 3 over $\Q$, where $\alpha$ is any root of $p(x)$.  Thus the degree of $E$ will be either $3$ or $6$.  This means that the size of $G$ will either be 3 or 6, but in either case, by Cauchy's theorem, since $3$ divides the order of the group and 3 is prime, there must be an element of order $3$.
		\part Note that $p(x)$ only has one real root, so the other two roots are complex.  This means that the complex conjugation automorphism will switch two roots, so this is a non-identity element of $G$.  But since complex conjugation is its own inverse, we see that this element has order 2.
		\part No matter what $G$ is (it will be $S_3$ or $\Z_6$), there is a normal subgroup $H$.  Using the Galois correspondence, this means that the fixfield of $H$ will be a normal extension, i.e., an extension which is the splitting field for some polynomial.
		\part We get this again by the Galois correspondence.  One of the intermediate fields between $\Q$ and $E$ is $\Q(\alpha)$ where $\alpha$ is the real root of $p(x)$.  Since this field is purely real, we know that it is NOT a splitting field.  Therefore its fixer is a subgroup of $G$ that is not normal.  But $\Z_6$ is abelian, so all its subgroups are normal.
		\part In fact, $p(x)$ is solvable by radicals, as are all degree 3 polynomials (Cardano's formula).  The key here is that $S_3$ is a solvable group, since $S_3 \supset A_3 \subset \{(1)\}$ is a normal series in which each quotient group is abelian.
	\end{parts}
\end{answer}




\end{questions}


\Writetofile{\ansfilename}{
\protect\end{questions}
  
\protect\end{document}}
\Closesolutionfile{\ansfilename}

\end{document}


