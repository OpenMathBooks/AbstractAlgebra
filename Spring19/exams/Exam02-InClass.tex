\documentclass[11pt]{exam}

\usepackage{amssymb, amsmath, amsthm, mathrsfs, multicol, graphicx} 
\usepackage{tikz}

\def\d{\displaystyle}
\def\?{\reflectbox{?}}
\def\b#1{\mathbf{#1}}
\def\f#1{\mathfrak #1}
\def\c#1{\mathcal #1}
\def\s#1{\mathscr #1}
\def\r#1{\mathrm{#1}}
\def\N{\mathbb N}
\def\Z{\mathbb Z}
\def\Q{\mathbb Q}
\def\R{\mathbb R}
\def\C{\mathbb C}
\def\F{\mathbb F}
\def\A{\mathbb A}
\def\X{\mathbb X}
\def\E{\mathbb E}
\def\O{\mathbb O}
\def\FR{\mathscr{F(\R)}}
\def\pow{\mathscr P}
\def\inv{^{-1}}
\def\nrml{\triangleleft}
\def\st{:}
\def\~{\widetilde}
\def\rem{\mathcal R}
\def\iff{\leftrightarrow}
\def\Iff{\Leftrightarrow}
\def\and{\wedge}
\def\And{\bigwedge}
\def\AAnd{\d\bigwedge\mkern-18mu\bigwedge}
\def\Vee{\bigvee}
\def\VVee{\d\Vee\mkern-18mu\Vee}
\def\imp{\rightarrow}
\def\Imp{\Rightarrow}
\def\Fi{\Leftarrow}

\def\={\equiv}
\def\var{\mbox{var}}
\def\mod{\mbox{Mod}}
\def\Th{\mbox{Th}}
\def\sat{\mbox{Sat}}
\def\con{\mbox{Con}}
\def\bmodels{=\joinrel\mathrel|}
\def\iffmodels{\bmodels\models}
\def\dbland{\bigwedge \!\!\bigwedge}
\def\dom{\mbox{dom}}
\def\rng{\mbox{range}}
\DeclareMathOperator{\wgt}{wgt}
\DeclareMathOperator{\Gal}{Gal}
\DeclareMathOperator{\ord}{ord}

\def\bar{\overline}

%\pointname{pts}
\pointsinmargin
\marginpointname{pts}
\marginbonuspointname{ bns pts}

\addpoints
\pagestyle{headandfoot}
%\printanswers

\firstpageheader{Math 322}{\bf\large Exam 2 -- In Class}{April 15, 2019}
\runningfooter{}{\thepage}{}
\extrafootheight{-.45in}



\begin{document}
%space for name
\noindent {\large\bf Name:} \underline{\hspace{2.5in}}
\vskip 1em

%\noindent{\bf Instructions:} Answer each of the following questions, and make sure you SHOW ALL YOUR WORK!  Answers without supporting work will be counted as incorrect.  When asked to explain or prove your answers, use complete English sentences.





\begin{questions}



  
\uplevel{For the problems on this page, let $G$ be a finite group containing an element $a$.}
\question[6] Prove that if $\ord(a) = n$ and $a^k = e$, then $k$ is a multiple of $n$.  Your proof should use the division algorithm.
\begin{solution}
\begin{proof}
Suppose $\ord(a) = n$ and $a^k = e$.  Since $\ord(a) = n$, we know that $a^n = e$ and that no positive number smaller than $n$ has this property.  Now apply the division algorithm to $n$ and $k$.  We get $k = qn + r$ where $0 \le r < n$.  We also have
\[a^k = a^{qn+r} = (a^n)^q\cdot a^r\]
But $a^k = e$ and $(a^n)^q = e^q = e$ so this says
\[a^r = e.\]
The only way this can happen is if $r = 0$, so we get that $k = qn$ as required.
\end{proof}
\end{solution}



\vfill

\question[6] What can you say about the orders of $a^m$?  Use the previous question to prove that for any $m$ we have that $\ord(a^m)$ is a divisor of $\ord(a)$.  (Another way to say this is that $\ord(a)$ is a multiple of $\ord(a^m)$.)  
\begin{solution}
\begin{proof}
Let $\ord(a) = k$ and $\ord(a^m) = n$.  We have $(a^m)^k = (a^k)^m = e^m = e$.  Thus by the previous question, $k$ is a multiple of $\ord(a^m)$.
\end{proof}

\end{solution}

\vfill


\newpage

\question[6] Find all abelian groups of order $1125 = 3^2 \cdot 5^3$ which contain at least one element of order 25.  Briefly explain how you know you have them all.

\begin{solution}
By the Fundamental Theorem of Finite Abelian Groups, we know that these groups can be written as direct product of cyclic $p$-groups.  Each abelian group will be isomorphic to one of the following:
\[\Z_{3^2} \times \Z_{5^3} \qquad \Z_{3^2} \times \Z_{5^2}\times \Z_5\qquad \Z_{3^2} \times \Z_{5}\times \Z_5\times \Z_5\]
\[\Z_{3}\times \Z_3 \times \Z_{5^3} \qquad \Z_{3}\times \Z_3 \times \Z_{5^2}\times \Z_5\qquad \Z_{3}\times \Z_3 \times \Z_{5}\times \Z_5\times \Z_5\]

Only 4 of these have elements of order 25:

\[\Z_{3^2} \times \Z_{5^3} \qquad \Z_{3^2} \times \Z_{5^2}\times \Z_5 \qquad \Z_{3}\times \Z_3 \times \Z_{5^3} \qquad \Z_{3}\times \Z_3 \times \Z_{5^2}\times \Z_5 \]

The two we excluded do not have elements of order 25 since when constructing the direct products we considered the $5$-groups.  If we had an element of order 25, we would have taken the subgroup generated by such an element and that would have given us $\Z_{25}$ as part of the direct product.

Note that $\Z_{125}$ does have an element of order 25, namely $5 \in \Z_{125}$.
\end{solution}

\vfill
\question[4] Of the groups (of order 1125) described in the previous question, how many are cyclic?  Explain.
\begin{solution}
Only one: $\Z_{9}\times \Z_{125}$.  Since $9$ and $125$ are relatively prime, this is isomorphic to $\Z_{1125}$, which is cyclic (alternatively, the element $(1,1)$ generates).  Note that all the other groups still have order $1125$ but are not isomorphic to this one (they are different!) so they cannot be cyclic (there is only one cyclic group of each order).
\end{solution}

\vfill

\newpage

\question[16] For each of the statements below, decide whether they are \textbf{TRUE} or \textbf{FALSE}.  Then justify your choices either with a brief explanation (if true) or counterexample (if false).
\begin{parts}
\part Every element in $S_8$ can be written as the product of 8 transpositions.
\begin{solution}
False.  For example, the element $(1234) = (12)(23)(34)$ can be written as an odd number of permutations, so every way to write it will use an odd number of permutations.
\end{solution}
\vfill
\part Every element of $A_8$ can be written as the product of 8 transpositions.
\begin{solution}
True.  $A_8$ contains all the even permutations in $S_8$.  Even permutations like $(12)(23)$ can still be written as 8 transpositions by adding on $(12)(12)$ enough times to bring the total number up to 8.  We must simply check that no element in $A_8$ requires an even number of transpositions \emph{larger} than $8$.  Every element in $A_8$ can be written as the product of disjoint cycles.  These could be a single cycle of length 7 (which gives exactly 8 transpositions) or less, as a 2-cycle and a 6 (or smaller) cycle (giving 8 transpositions), a 3-cycle and a 5-cycle (giving 8 transpositions), etc.
\end{solution}
\vfill
\part For any group $G$, all subnormal series of $G$ are the same length.
\begin{solution}
False.  For example, $\Z_{12} \supset \langle 6\rangle \supset \{0\}$ and $\Z_{12} \supset \langle 2\rangle \supset \langle 6 \rangle \supset \{0\}$ are both subnormal series, but have different length.  
However, all \emph{composition} series have the same length by the Jordan-H\"older theorem.
\end{solution}
\vfill
% \part For any group $G$, all composition series of $G$ are the same length.
% \begin{solution}
%   True.  This is implied by the Jordan-H\"older theorem.
% \end{solution}
% \part For any group $G$, if $m$ divides the order of $G$, then there is an element of $G$ with order $m$.
% \begin{solution}
% False.  This would be true if $m$ were prime (by Cauchy's theorem) but in general this is false.  For example, $S_5$ has order 120, but there is no element with order larger than 6.  Note that if the statement were true, then all groups would be cyclic.
% \end{solution}
\part If $a$ has order 5 then the cyclic group $\langle a\rangle$ is isomorphic to a subgroup of $S_5$.
\begin{solution}
  True, by Cayley's theorem, since $\langle a \rangle$ has 5 elements.
\end{solution}
\vfill

\end{parts}

\newpage 

\question[12] Suppose $p(x)$ is a degree 7 polynomial that is irreducible over $\Q$.  Let $E$ be the splitting field for $p(x)$, and let $G = \Gal(E/\Q)$ be the Galois group.

\begin{parts}
  \part Carefully explain (citing any appropriate theorem) why $G$ has an element of order 7.
  \vfill
  \part Suppose $p(x)$ has at least one non-real complex root.  Carefully explain why $G$ has an element of order 2.
  \vfill
\end{parts}
\end{questions}




\end{document}


