\documentclass[11pt]{exam}

\usepackage{amssymb, amsmath, amsthm, mathrsfs, multicol, graphicx} 
\usepackage{tikz}

\def\d{\displaystyle}
\def\?{\reflectbox{?}}
\def\inv{^{-1}}
\def\b#1{\mathbf{#1}}
\def\f#1{\mathfrak #1}
\def\c#1{\mathcal #1}
\def\s#1{\mathscr #1}
\def\r#1{\mathrm{#1}}
\def\N{\mathbb N}
\def\Z{\mathbb Z}
\def\Q{\mathbb Q}
\def\R{\mathbb R}
\def\C{\mathbb C}
\def\F{\mathbb F}
\def\A{\mathbb A}
\def\X{\mathbb X}
\def\E{\mathbb E}
\def\O{\mathbb O}
\def\FR{\mathscr{F(\R)}}
\def\pow{\mathscr P}
\def\inv{^{-1}}
\def\nrml{\triangleleft}
\def\st{:}
\def\~{\widetilde}
\def\rem{\mathcal R}
\def\iff{\leftrightarrow}
\def\Iff{\Leftrightarrow}
\def\and{\wedge}
\def\And{\bigwedge}
\def\AAnd{\d\bigwedge\mkern-18mu\bigwedge}
\def\Vee{\bigvee}
\def\VVee{\d\Vee\mkern-18mu\Vee}
\def\imp{\rightarrow}
\def\Imp{\Rightarrow}
\def\Fi{\Leftarrow}

\def\={\equiv}
\def\var{\mbox{var}}
\def\mod{\mbox{Mod}}
\def\Th{\mbox{Th}}
\def\sat{\mbox{Sat}}
\def\con{\mbox{Con}}
\def\bmodels{=\joinrel\mathrel|}
\def\iffmodels{\bmodels\models}
\def\dbland{\bigwedge \!\!\bigwedge}
\def\dom{\mbox{dom}}
\def\rng{\mbox{range}}
\DeclareMathOperator{\wgt}{wgt}
\DeclareMathOperator{\Gal}{Gal}
\DeclareMathOperator{\ord}{ord}


\def\bar{\overline}

%\pointname{pts}
\pointsinmargin
\marginpointname{pts}
\marginbonuspointname{bn-pts}

\addpoints
\pagestyle{headandfoot}
%\printanswers

\firstpageheader{Math 322}{\bf\large Exam 2 -- Take Home}{Spring 2019}
\runningfooter{}{\thepage}{}
\extrafootheight{-.45in}



\begin{document}
%space for name
\noindent {\large\bf Name:} \underline{\hspace{2.5in}}
\vskip 1em

\noindent{\bf Instructions:} This is the take-home portion of the second exam.  Here are my expectations:

\begin{itemize}
\item WORK ALONE!  You may \underline{not} collaborate or discuss problems with other students, either in or outside of this class.  Also do not discuss with tutors, significant others, parents, kids, etc.  Cats and dogs are okay, but only if they have not taken an abstract algebra class. If you need clarification on a problem, ask me.

\item You may use your notes (including notes from last semester) or the textbook, but only notes you have taken in this class (or in Algebra I) and only the assigned textbook from this class.  This is intended only for you to refresh your memory if you forget a definition, not for you to copy proofs (or even the style of proof) from your notes.  Alternatively, if you do not remember a definition, send me an email.

\item Other than your notes and textbook, do not use any outside sources. In particular, absolutely NO INTERNET.  

\item This is not a timed exam, and you may take as much time on it as you like.  However, I do not intend for you to spend more than 3 hours total working on the exam.

\item You should write up all solutions neatly on your own paper and staple this sheet to your solutions.  Clearly number each problem, and put the problems in the usual numerical order (no non-identity permutations please).

\item As always, you must show all your work to receive credit, and explanations and proofs should be written out in complete English sentences.  A page of just equations and calculations will probably receive no credit.

\item \textbf{Due Monday, April 15.}
\end{itemize}


\vfill

\centerline{Have Fun!}

\vskip 1em

\hrulefill
\vskip 2em

\newpage

\begin{questions}
%Takehome:
% \question[12] The group $S_5$ is not cyclic, but it turns out that $S_5$ can be generated by just two elements.  In fact, you can take any $5$-cycle and any $2$-cycle.  Prove this by completing the following:
% \begin{parts}
% \part Explain why the set of all transpositions generates $S_5$.
% \begin{solution}
% We know that every element in $S_5$ can be written as the product of transpositions.  Thus if we have all the transpositions, we have all elements in $S_5$.
% \end{solution}
% \part Explain why the set $\{(12),(13),(14),(15)\}$ generates all the transpositions (and thus all of $S_5$).
% \begin{solution}
% Take conjugates: $(12)(13)(12) = (23)$.  In fact, $(1i)(1j)(1i) = (ij)$ so we can get any transposition this way.
% \end{solution}
% \part Explain why any 5-cycle and any transposition together generate $\{(12), (13), (14), (15)\}$ (and thus all of $S_5$).  It might be helpful to use a specific example at first, but your explanation should use a general $5$-cycle and transposition.
% \part Show that this does not work in $S_6$.  That is, find a transposition and a 6-cycle which do NOT generate $S_6$.
% \end{parts}





\question[12] Both Lagrange's theorem and Cauchy's theorem deal with the relationship between the size of a group and the order of its elements. 
\begin{parts}
\part Explain the difference between the theorems in general terms and by using $S_7$ as an example.  Your explanation should include what we can and cannot conclude from each theorem about $S_7$.
\begin{solution}
Lagrange's theorem says that if $a$ is an element of a group, then the order of $a$ divides the order of the group.  Cauchy's theorem says that if a \emph{prime} number divides the order of a group, then there is an element of that order.  So Cauchy's theorem is a partial converse to Lagrange's theorem (but only for primes). 

In $S_7$, which has order $7! = 5040$, Lagrange's theorem tells us that every element has order dividing $5040$.  In particular, there is no element with order 11, since $11$ does not divide $7!$.  However, Lagrange's theorem cannot say that there is an element of order 8, even though $8$ does divide $5040$ (in fact, there is no element of order 8).  Cauchy's theorem tells us that for sure there are elements of orders 2, 3, 5, and 7.  But this is all.  We cannot use Cauchy to say there are elements of order 6 (even though there are) or that there are no elements of order 8 (even though there are none).

\end{solution}


\part Which theorem would allow you to prove that if a group contained only elements that had order some power of $2$, then the order of the group must be a power of 2?  You might want to consider contrapositives.  Then \textbf{give the proof}.

\begin{solution}
Lagrange's theorem is not helpful because elements have an order that is a power of 2 can occur in groups whose order is not a power of 2 (as long as it is a multiple of a power of 2).  Cauchy's theorem is helpful though.  If the group did not have order some power of 2, then another prime would divide the order of the group, and then Cauchy's theorem would give us an element that had order that other prime.

\end{solution}

\part We have previously proved that if $G$ is a cyclic group of order $n$, then for every factor $m$ of $n$, $G$ contains elements of order $m$.  Explain why neither Lagrange's theorem nor Cauchy's theorem can be used to prove this.

\begin{solution}
If $n$ is not prime, then it will have divisors (including itself) that are not prime, so Cauchy's theorem does not apply.  On the other hand, Lagrange's theorem tells us that $G$ cannot have orders other than factors of $n$, but does not tell us if every factor of $n$ is the order of some element.
\end{solution}

\end{parts}


\question[16] Consider groups generated by two elements $G = \langle a, b \rangle$.  Note that these groups \emph{could} still be cyclic, depending on the relationship between $a$ and $b$.
\begin{parts}
\part Suppose $G = \langle a, b \rangle$ is abelian with $\ord(a) = 4$ and $\ord(b) = 5$.  List all the elements of $G$.  What familiar group is $G$ isomorphic to?  Justify your answer using internal direct products.  Hint: one of the elements you list should be either $a^2b^3$ or $b^3a^2$ (these are the same element).

\begin{solution}
We have $\{e, a, a^2, a^3, b, ba, ba^2, ba^3, b^2, b^2a, b^2a^2, b^2a^3, b^3, b^3a, b^3a^2, b^3a^3, b^4, b^4a, b^4a^2, b^4a^3\}$.  This is everything since $a$ and $b$ commute.  We can view $G$ as the internal direct product of $\langle a \rangle$ and $\langle b \rangle$.  Since $\langle a \rangle \cong \Z_4$ and $\langle b \rangle \cong \Z_5$, we see that $G \cong \Z_4\times \Z_5 \cong \Z_{20}$ (since 4 and 5 are relatively prime).  

Note that in this example, $G$ is actually cyclic, generated by $ab$.
\end{solution}

\part Find an example of an abelian group $G = \langle a, b\rangle$ where the order of $G$ is strictly less than $\ord(a)\cdot \ord(b)$.  (Do not assume the orders are the same as the previous part.) Explain why your example works.

\begin{solution}
In general, if $b$ is a power of $a$, then adding $b$ to the list of generators will not change the order, and we will get a group of size $\ord(a)$.  For example, consider $a = 1$ and $b = 3$ under the operation of addition mod 6.  This gives $\Z_6$ even though $\ord(1) = 6$ and $\ord(3) = 2$.
\end{solution}

\part Prove that if $G = \langle a, b \rangle$ is abelian, then $|G| \le \ord(a)\cdot \ord(b)$.

\begin{solution}
Every element will be of the form $a^jb^k$: we can always rearrange any product of $a$'s and $b$'s so all the $a$'s come first.  This is where we use the fact that $G$ is abelian.  There are exactly $\ord(a)$ different choices for $j$ and exactly $\ord(b)$ choices for $k$, giving $\ord(a)\ord(b)$ choices for elements of this form.  If all of these are distinct, then $|G| = \ord(a)\ord(b)$, but of course some of these might be repeats, as seen in the example above.
\end{solution}

\part Give an example of a non-abelian group $G = \langle a,b\rangle$ that shows the previous part is not true for groups in general. Explain why your example works and what goes wrong when you try to use the proof you gave in the previous part.
\begin{solution}
For example, $S_5 = \langle (12), (12345)\rangle$.  We have $\ord((12)) = 2$ and $\ord((12345)) = 5$ but $|S_5| = 120 \not\le 10$.  The problem is that here $(12)(12345)(12) \ne (12345)$ for example.  Since we cannot rearrange the generators to group them together, we get more elements.  In fact, it is possible to get an infinite group using two generators both with finite order.
\end{solution}


\end{parts}




\question[16] Consider the group $\Z_{45}$.
\begin{parts}
\part Give an example of a subnormal series which is not a composition series.  Explain why your example works.
\begin{solution}
$\Z_{45} \supset \langle 15\rangle \{0\}$ is such a subnormal series.  It is subnormal since every subgroup is normal in its predecessor ($\Z_{45}$ is abelian so all subgroups are normal).  However, $\Z_{45}/\langle 15\rangle \cong \Z_{15}$ which is not simple.
\end{solution}


\part Find a \emph{refinement} of the series you gave above which is a composition series.  That is, show how to extend your series into a composition series.

\begin{solution}
We can add a subgroup between $\Z_{45}$ and $\langle 15\rangle$ since that quotient group is not simple.  We get
\[\Z_{45} \supset \langle 5 \rangle \supset \langle 15 \rangle \supset \{0\}\]
This is a composition series now since the quotient groups, reading from left to right are $\Z_5$, $\Z_3$ and $\Z_3$, all simple groups.
\end{solution}

\part Find \emph{all} other composition series and briefly explain how you know you have them all.

\begin{solution}
In addition to the one found above, we have
\[\Z_{45} \supset \langle 3 \rangle \supset \langle 15 \rangle \supset \{0\}\]
\[\Z_{45} \supset \langle 3 \rangle \supset \langle 9 \rangle \supset \{0\}\]
This is all of them.  By the Jordan-H\"older theorem, all composition series must be isomorphic, which means they have the same quotient groups just possibly in a different order.  The quotient groups must be $\Z_3$, $\Z_3$ and $\Z_5$, and these can come in one of three orders (based on where the $\Z_5$ falls).  Once we have the quotient groups, there is only one choice for subgroup (in this case) that gives that quotient group.
\end{solution}

\part If $p(x)$ is a polynomial whose splitting field has Galois group over $\Q$ isomorphic to $\Z_{45}$, will the roots of $p(x)$ be expressible in terms of $n$th roots and field operations?  Briefly explain (you can cite a result we discussed in class).

\begin{solution}
Yes.  Since the Galois group for $p(x)$ is solvable, we know the polynomial will be solvable by radicals.
\end{solution}

\end{parts}

\question[6] Suppose $E$ is a splitting field whose Galois group over $\Q$ is isomorphic to $\Z_{45}$.  How many different intermediate fields are there between $\Q$ and $E$?  What degree extensions are these?  Justify your answers.  You may refer to the work you did on the previous problem (part (c) in particular) as well as reference the Fundamental Theorem of Galois Theory.

\bonusquestion[10] Bonus: Let $E$ be as in the previous problem, and $p(x)$ a polynomial for which $E$ is the splitting field.  What can you say about $p(x)$?  What could it's degree be?  Could it be irreducible?  Can you give an example of such a polynomial?  Justify your answers.  (The more you can say, the more bonus points you will get.)

\end{questions}




\end{document}


