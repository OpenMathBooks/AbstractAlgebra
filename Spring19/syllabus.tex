\documentclass[12pt,letterpaper]{article}


\usepackage{fullpage, graphicx, url}
\usepackage{enumerate}
\usepackage{multicol}


\pagestyle{empty}
\thispagestyle{empty}


\begin{document}

\begin{center}\textbf{\Large Introduction to Abstract Algebra II}

\textbf{Math 322 Spring 2019 (3 credits)}
\end{center}
\vskip 2ex

\noindent\textbf{Professor}: Oscar Levin, Ph.D,~~ Ross 2239G,~~ 351-2380,~~ \url{oscar.levin@unco.edu}

\noindent\textbf{Office Hours}: M/W/F 10:00-11:00.  Also other times by appointment.

\noindent\textbf{Website}: through Canvas.

\noindent\textbf{Textbook}: We will use a customized version of \emph{Abstract Algebra: Theory and Applications} by Tom Judson, 2018 edition.  See Canvas for the link to the online notes.

\vskip 2 ex

Welcome to what promises to be yet another exciting and fun filled semester of Abstract Algebra!  I know you are all eager to get started, but please take a few moments to glance at this syllabus.
\vskip 2ex

\noindent
\textbf{Prerequisite}: MATH 321 with a grade of C or better
\vskip 2 ex

\noindent
\textbf{Course Description}: A continuation of MATH 321. Topics will include: groups, rings, fields and Galois theory, as well as applications of these topics to other areas in mathematics.

\vskip 2ex
\noindent
\textbf{Course Objectives}: Students successfully completing this class will have mastered a basic understanding of the concepts in abstract algebra, generally, and group, ring and field theory, specifically.  Additionally, students will improve their ability at reading, writing and validating proofs. 

\vskip 2 ex

\noindent
\textbf{Outline of Course Content}: Here is a list of topics grouped not chronologically, but topically.
\begin{enumerate}\itemsep1pt \parskip0pt \parsep0pt
\item Advanced group theory, the structure of finite groups, permutations.
\item Group actions and applications to combinatorics.
\item Further exploration of rings, polynomials, homomorphisms and quotient rings.
\item Field extensions, basic Galois theory, and applications to geometry and polynomials.
\end{enumerate}

\vskip 2ex

\noindent
\textbf{Method of Evaluation}: Your final letter grade will be calculated as follows:
\vskip 1ex

\begin{tabular}{llcll}
Homework: & 25\% & \qquad & Quizzes: & 10\% \\
Exams: & 20\% each & & Final Exam: & 20\% \\
Participation and Effort: & 5\%
\end{tabular}
\vskip 1ex
Grades will be assigned according to the following scale:
\begin{center}
\begin{tabular}{||c|c|c|c|c|c||}\hline
93-100\%: A& 90-92\%: A-  & 87-89\%: B+ & 83-86\%: B & 80-82\%: B- & 77-79\%: C+ \\
73-76\%: C & 70-72\%: C- & 67-69\%: D+ & 63-66\%: D & 60-62\%: D- & $\leq 59\%$: F \\ \hline
\end{tabular}
\end{center}

\vskip 2 ex

\noindent\textbf{Course Requirements}:
\vskip 1ex
\underline{Exams}: There will be two midterm exams and a cumulative final.  The midterm exams will contain both an in-class and take-home portion (the take-home portion will be due the following class period).  Midterms are \textit{tentatively} scheduled for the following dates:

\begin{tabular}{ll}
 Exam 1: & Friday, February 15 (take-home part due Monday, February 18) \\

 Exam 2: & Friday, April 12 (take-home part due Monday, April 15)
\end{tabular}

The cumulative final exam will be on \textbf{Friday, May 3 at 1:30pm}.
\vskip 1 ex

\underline{Quizzes}: There will be two types of quizzes: frequent online reading quizzes (on Canvas) and occasional short (10 minutes) in-class quizzes.  These can cover any material from previous lectures, in-class activities or the practice homework problems.  Reading quizzes will always be posted in Canvas; you should check for one for every class period (they will be due two hours before class).  In-class quizzes will rarely be announced ahead of time and you should be prepared for a quiz on any given day of class. These quizzes will allow you to check yourself on some basic problems as we move through the semester, so that you will not be surprised when you get to the exams, and to ensure you are keeping up with the material. Missed quizzes may NOT be made up under any circumstances.

\vskip 1 ex
\underline{Homework}: Homework will be assigned for each topic we cover.  There are two basic types of homework assignments.  First, there will be a reasonably large number of practice problems for you to try, but not turn in.  Instead, you should come to class prepared to discuss these exercises in small or larger groups, and demonstrate mastery of them on quizzes.  Second, there will be a smaller number of problems each week which you will turn in on paper. These should be written out neatly and include the correct answer, your work, and most importantly, an explanation of why your answer is correct.  You should do the problems as soon as possible, but the homework from the previous week's material will be due each Wednesday.  Both types of assignments will always be listed on Canvas so check it regularly.

\vskip 1 ex

\underline{Class-work}:  Class periods will be a mix of lecture, discussion and investigation, with an emphasis on the latter two.  Come to class ready to actively do some math.  Outside of class, I encourage you to work in small groups as well.  Actively participating in your own learning, as well as helping your classmates, is the best way to succeed in the course.
\vskip 2 ex

\noindent
\textbf{Attendance Policy}: You are expected to attend every class period.

\vskip 2 ex

\noindent\textbf{Makeup Policy}: In general, missed exams may not be made up and homework may not be turned in late.  Exceptions will be made only in \emph{very} extreme cases.  Please contact me well in advance whenever possible if you need me to consider such an exception.  Note, missed quizzes cannot be made up under any circumstances.
\vskip 2ex


\noindent\textbf{Classroom Policies}: Don't be rude.  Please be considerate of your fellow classmates and do not act in a disruptive manor.  Turn off your cell phones and mp3 players before coming to class and keep them put away, arrive on time, and do not pack up your things before the end of class.  When working in groups, try your hardest to keep the conversation on the mathematics at hand.  If you need to leave the room for any reason (like using the restroom) please do so as quietly as possible.  Since your cell phones should be off, this of course means \textbf{no texting}.

\vskip 2 ex

\noindent\textbf{Statement of Academic Integrity}: Don't cheat!  It is expected that members of this class will observe strict policies of academic honesty.  In particular, you are expected to solve homework problems by yourself or together with your group, and not find solutions online.  In general, UNC's policies and recommendations for academic misconduct will be followed. For additional information, please see the Student Code of Conduct at the Dean of Student's website \url{http://www.unco.edu/dos/Conduct/codeofconduct.html}. In the case of academic appeals, university procedures will be followed. For information on academic appeals, see \url{http://www.unco.edu/regrec/Current%20Students/AcademicAppeals.html}.

\vskip 2 ex

\noindent\textbf{Disability Statement}: It is the policy and practice of the University of Northern Colorado to create inclusive learning environments.  If there are aspects of the instruction or design of this course that present barriers to your inclusion or to an accurate assessment of your achievement (e.g. time-limited exams, inaccessible web content, use of videos without captions), please communicate this with your professor and contact Disability Support Services (DSS) to request accommodations.  Office: (970) 351-2289, Michener Library L-80. Students can learn more about the accommodation process at \url{http://www.unco.edu/disability-support-services/}.

\vskip 2ex

\noindent\textbf{Suggestions for a Successful Semester}:
\begin{enumerate}\itemsep1pt \parskip0pt \parsep0pt
\item Your \textbf{JOB} as a student of mathematics is to \textbf{ask questions}.  This can be difficult but it is an important skill that will serve you well.  Use this class as a safe place to practice.  My promise: \underline{any} question you ask will only ever \underline{improve} my opinion of you.
\item Think critically! Don't believe something just because I tell you that it's true. Always
ask yourself if you have good reason to believe it.
\item Do all the practice and assigned homework, as soon as possible.  Practice, practice, etc.
\item Challenge yourself.  Some topics we study might come easy to you, others not.  You should look for these challenges, work hard, and overcome them.  You are here to learn, not to demonstrate what you already know.
\setcounter{enumi}{5}
\item Don't skip numbers.
\item Work with others.  We will do a lot of group work in class.  There is no reason you can't continue to work with your new friends on the homework and when studying for exams.  Teaching each other mathematics is the best way to learn it.
\item Don't ride a bike without a helmet.  Especially stationary bikes!
\item If you need help, come see me in my office during the hours listed above or make an
appointment with me for some other time. My door is always figuratively open when it's literally open.
\item Most importantly, if you don't understand something: ASK!  See suggestion number 1.  You are one of only a handful of students in the class, so please please please interrupt when something is unclear.
\end{enumerate}




\end{document}
