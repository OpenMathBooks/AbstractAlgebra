
\documentclass[12pt]{exam}

\usepackage{amssymb, amsmath, amsthm, mathrsfs, multicol, graphicx}
\usepackage[top=.25in, bottom=1in, left=1in, right=1in]{geometry}
% \usepackage{fancyhdr}
\usepackage{url}
\usepackage{enumitem}
\usepackage{pdfpages}
\usepackage{tikz}
\usetikzlibrary{positioning,matrix,arrows}


\usepackage{xcolor}

\pagecolor[rgb]{0,0,0}
\color[rgb]{1,1,1}

\pagestyle{empty}
% \include{macros-descrete}


\setlength\parindent{0pt}

%%% macros %%%%%
\newcommand{\ex}{\noindent\textbf{Ex:} }

\def\d{\displaystyle}
% \def\uplevel#1{\end{itemize}#1\begin{itemize}}
\def\deg{^\circ}
\def\st{~:~}

\def\imp{\rightarrow}
\def\Imp{\Rightarrow}
\def\iff{\Leftrightarrow}
\def\Iff{\Leftrightarrow}
\def\land{\wedge}



\def\N{\mathbb N}
\def\Z{\mathbb Z}
\def\R{\mathbb R}
\def\Q{\mathbb Q}
\def\P{\mathbb P}
\def\E{\mathbb E}
\def\O{\mathbb O}
\def\F{\mathbb F}
\def\pow{\mathcal{P}}
\def\b{\mathbf}
\def\st{~|~}
\def\bar{\overline}
\def\inv{^{-1}}

\newcommand{\vtx}[2]{node[fill,circle,inner sep=0pt, minimum size=5pt,label=#1:#2]{}}
\newcommand{\va}[1]{\vtx{above}{#1}}
\newcommand{\vb}[1]{\vtx{below}{#1}}
\newcommand{\vr}[1]{\vtx{right}{#1}}
\newcommand{\vl}[1]{\vtx{left}{#1}}
\renewcommand{\v}{\vtx{above}{}}

%headers and footers:


%opening

\begin{document}

\begin{questions}


  \question The group $D_3$ of symmetries of the triangle is isomorphic to $S_3$.  But by Cayley's theorem, the group is also isomorphic to a {\em subgroup} of $S_6$.  Find such a subgroup (using the proof of Cayley's theorem).
  
  \clearpage
  
  
  
  \question The identity can be written as $\varepsilon = (13)(24)(35)(14)(12)(15)(34)(45)$.
  Mimic the proof that $\varepsilon$ must be even and show how to eliminate $x = 5$ from the product of transpositions and write $\varepsilon$ as the product of 2 fewer transpositions in the process.  Show all intermediate steps.
  
  \clearpage
  
  
  
  
  \question Suppose the group $G$ has subnormal series 
  \[G \supset H \supset \{e\}\]
  and that $G/H \cong \Z_{10}$.  Assume also that $H$ is simple.
  \begin{parts}
  \part Explain how we know that the above series is not a composition series.

  \vfill
  \part Explain how we could find two different composition series for $G$.

  \vfill
  \part Prove that if $H$ is abelian, then $G$ is solvable.

  \vfill
  \part If $G$ happens to be the Galois group for some field $E$ over $\Q$, 
  what can you say about subfields of $E$?
  \vfill
  \end{parts}
  
    \clearpage
  
  
  \question Consider the polynomial $p(x) = x^3 + 5x^2 - 10x + 15$.  Let $E$ be the splitting field for $p(x)$ and $G$ be the Galois group of $E$ over $\Q$.
  \begin{parts}
    \part Prove that $G$ contains an element of order 3.

    \vfill
    \part Prove that $G$ contains an element of order 2.

    \vfill
    \part Explain how we know that there is a intermediate field $I$ strictly 
    between $\Q$ and $E$ that is the splitting field for a polynomial.  What can you say about this field?
    \vfill
    \part Explain how you know that $G \cong S_3$ and not to $\Z_6$.

    \vfill
    \part Does the argument above prove that $p(x)$ is not solvable by 
    radicals?  Is $p(x)$ solvable by radicals?
    \vfill
  \end{parts}
  
  \clearpage
  
  \question Consider the number $n = 1643 = 31\cdot 53$.  
  \begin{parts}
    \part What is $42^{1560}$ congruent to modulo 1643?  Explain, using group 
    theory.  What if we replaced 42 with another number?  
    \vfill
    \part Note that $E = 7$ is relatively prime to $1643$. Find an integer $D$ 
    such that $(a^{7})^D \equiv a  \pmod{1643}$ for any $a$ relatively prime to $1643$. Explain how you know your $D$ works. 
    \vfill
  \end{parts}
  
  \clearpage
  
  \question What is the difference between an inner direct product and an external direct product?  Illustrate with an example.
  
  \clearpage
  
  
  \question I'm thinking of an abelian group that contains elements of order $9$ but not of order $27$, and elements of order $2$ and $5$ but not of orders $4$ or $25$.  Further, there is not other prime number $p$ (other than 2, 3, or 5) such that there is an element of order $p$. Which of the following can you deduce about my group?  Which must be true, which can't be true, and which might or might not be true.
  \begin{parts}
    \part There are elements of order 3.

    \vfill
    \part There are elements of order 10.

    \vfill
    \part The group is isomorphic to $\Z_2 \times \Z_{9}\times \Z_5$. 

    \vfill
    \part The group is cyclic.

    \vfill
  \end{parts}
  
  \clearpage
  
  \question Find all abelian groups of order 480.
  
  \clearpage
  
  
  
  
  \question Let $X = \{1,2,3,4,5,6\}$ and $G = \{(1), (12), (345), (354), (12)(345), (12)(354)\}$.  Find $X_g$, $G_x$ and $\mathcal O_x$ for each $g \in G$ and $x \in X$.  Then verify the orbit-stabilizer theorem and Burnside's theorem.
  
  \clearpage
  
  \question How many different ways could the vertices of an equilateral triangle be colored using three colors?
  
  \clearpage
  
  
  \question Use Burnside's theorem to explain why $\binom{7}{3} = \frac{1}{3!}P(7,3)$.  That is, why are there 6 times as many ways to make three scoop ice-cream cones chosen from 7 flavors as there are to make three scoop milkshakes (cones and shakes not allowing for repeated flavors).
  
  \clearpage
  \end{questions}

\end{document}