
\documentclass[12pt]{article}

\usepackage{amssymb, amsmath, amsthm, mathrsfs, multicol, graphicx}
\usepackage[top=.25in, bottom=1in, left=1in, right=1in]{geometry}
\usepackage{multicol}
\usepackage{fancyhdr}
\usepackage{url}
\usepackage{enumitem}
\usepackage{pdfpages}
\usepackage{tikz}
\usetikzlibrary{positioning,matrix,arrows}

\pagestyle{empty}
\include{macros-descrete}

\theoremstyle{plain}
\newtheorem*{theorem}{Theorem}
\newtheorem*{proposition}{Proposition}
\newtheorem*{lemma}{Lemma}
\newtheorem{problem}{Problem}
\newtheorem{question}{Question}
\newtheorem{answer}{Answer}

\theoremstyle{definition}
\newtheorem*{definition}{Def}
\newtheorem*{example}{Ex}

\theoremstyle{remark}
\newtheorem*{remark}{Remark}
\newtheorem*{solution}{Solution}

\setlength\parindent{0pt}

%%% macros %%%%%
\newcommand{\ex}{\noindent{\bf Ex: }}

\def\d{\displaystyle}
\def\uplevel#1{\end{itemize}#1\begin{itemize}}
\def\st{~:~}

\def\imp{\rightarrow}
\def\Imp{\Rightarrow}
\def\iff{\Leftrightarrow}
\def\Iff{\Leftrightarrow}
\def\land{\wedge}



\def\N{\mathbb N}
\def\Z{\mathbb Z}
\def\R{\mathbb R}
\def\Q{\mathbb Q}
\def\P{\mathbb P}
\def\E{\mathbb E}
\def\O{\mathbb O}
\def\F{\mathbb F}
\def\C{\mathbb C}
\def\F{\mathscr{F}}
\def\Rp{\R^{\mathrm{pos}}}
\def\FR{\mathscr{F}(\R)}
\def\CR{\mathscr{C}(\R)}
\def\DR{\mathscr{D}(\R)}
\def\pow{\mathcal{P}}
\def\b{\mathbf}
\def\v{\mathrm{\b v}}
\def\Im{\mathrm{Im}}
\def\bar{\overline}
\def\inv{^{-1}}
\def\lt{<}
\def\gt{>}
\DeclareMathOperator{\wgt}{wgt}
\DeclareMathOperator{\ord}{ord}
\DeclareMathOperator{\cis}{cis}
\DeclareMathOperator{\Aut}{Aut}
\DeclareMathOperator{\Gal}{Gal}
% \DeclareMathOperator{\deg}{deg}

\newcommand{\vtx}[2]{node[fill,circle,inner sep=0pt, minimum size=4pt,label=#1:#2]{}}
\newcommand{\va}[1]{\vtx{above}{#1}}
\newcommand{\vb}[1]{\vtx{below}{#1}}
\newcommand{\vr}[1]{\vtx{right}{#1}}
\newcommand{\vl}[1]{\vtx{left}{#1}}
\renewcommand{\v}{\vtx{above}{}}

%headers and footers:


%opening

\begin{document}

\Large
\begin{center}
 \textbf{\LARGE Orders and Euler's Theorem}\\
 Friday, March 27
\end{center}

\textbf{Last time}:  We conjectured that if \(p\) is prime, then for any \(a \lt p\) we have \(a^{p-1} \equiv 1 \pmod{p}\).  This led to a discussion of the \textbf{order} of elements.  Here is a summary of what we have so far:%
\begin{itemize}[label=\textbullet]
\item{}For any finite group \(G\) and any element \(g \in G\), we say the \textbf{order} of \(g\) is the least \(k\) such that \(a^k = e\) (the identity).%
\item{}We also noted that if \(\ord(g) = k\) then \(g, g^2, g^3,\ldots, g^k\) are distinct elements.  Why is this?%
\vskip 2in
% \item{}Suppose \(g^a = g^b\) with \(1 \le a \lt b \lt k\).  Then \(g^{a-a} = g^{b-a}\), but \(g^{a-a} = e\).  So that would mean that \(g^{b-a} = e\).  This is impossible since \(b-a\) is less than \(k\), and \(k\) was the least positive power of \(g\) that gives the identity.%
\item{}Since there are \(k\) distinct powers of \(g\), we have that the cyclic subgroup generated by \(g\), that is, \(\langle g \rangle\) contains exactly \(k\) elements.%
\item{}Thus the order of the element \(g\) is equal to the order of the cyclic subgroup generated by \(g\).%
\item{}But Lagrange's theorem tells us that the order of a subgroup must divide the order of the group.%
\item{}Thus the order of any element \(g \in G\) must divide the order of \(G\).%
\end{itemize}
\clearpage
%
\par
Now let's continue where we left off.%
\begin{itemize}[label=\textbullet]
\item{}Suppose \(\ord(g) = k\).  What is \(g^{nk}\) for any \(n\)?%
\vskip 3in

\item{}Now consider the group \(U(p)\) where \(p\) is prime.  This is the group of \emph{units} mod \(p\), which means \(\{1,2,3,\ldots, p-1\}\) (which is a consequence of Bezout's lemma).%
\vskip 3in
\item{}Thus in \(U(p)\) we have \(g^{p-1} = 1\) for all \(g \in U(p)\).%
\item{}Therefore \(a^{p-1} \equiv 1 \pmod{p}\).  This result is called\emph{Fermat's Little Theorem}.%
\end{itemize}
\clearpage
%
\par
Now what about the non-prime case?  Given \(n\), is there a number \(m\) such that \(a^m \equiv 1 \pmod{n}\) for all \(a\)?%
\begin{itemize}[label=\textbullet]
\vskip 2in
\item{}Working in groups again, we can consider \(U(n)\).  The elements of this group are precisely the \emph{units} of \(Z_n\), which means those elements relatively prime to \(n\).%
\vskip 2in
\item{}We will let \(\varphi(n)\) denote the \emph{number} of numbers less than \(n\) that are relatively prime to \(n\).  This is called the \textbf{Euler \(\varphi\) function}.%
\vskip 2in
\item{} What do we get if we repeat the same argument as we did for Fermat's Little Theorem?
\vskip 2in

This is known as Euler's Theorem.%
\end{itemize}
\clearpage
%
\par
For Euler's theorem to be useful, we need to understand how the \(\varphi\) function behaves.%
\begin{itemize}[label=\textbullet]
\item{}We know that \(\varphi(p) = p-1\) for any prime \(p\).  We also will define \(\varphi(1) = 1\) (because it will be useful to do so).%
\item{}The definition of \(\varphi(n)\) is: the number of positive integers less than \(n\) that are relatively prime to \(n\).  Find \(\varphi(n)\) by brute force for some non-prime values of \(n\).%
\item{}In particular, find \(\varphi(6)\), \(\varphi(10)\), \(\varphi(14)\), \(\varphi(15)\), and \(\varphi(21)\).  Note that each of these is the product of two primes.%
\item{}\(\varphi(4) = 2\), \(\varphi(6) = 2\), \(\varphi(8) = \)%
\end{itemize}
\end{document}