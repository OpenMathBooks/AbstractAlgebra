\documentclass[11pt]{exam}

\usepackage{amsmath, amssymb, multicol}
\usepackage{graphicx}
\usepackage{textcomp}
\usepackage{tikz}
\usetikzlibrary{arrows}

\def\d{\displaystyle}
\def\b{\mathbf}
\def\R{\mathbb{R}}
\def\Z{\mathbb{Z}}
\def\Q{\mathbb{Q}}
\def\N{\mathbb{N}}
\def\C{\mathbb{C}}
\def\st{~:~}
\def\bar{\overline}
\def\inv{^{-1}}


%\pointname{pts}
\pointsinmargin
\marginpointname{pts}
\addpoints
\pagestyle{head}
%\printanswers

\firstpageheader{Math 322}{\bf Group Actions}{Spring 2020}



\begin{document}

We will consider a new sort of operation: an interaction between a group and a set (which {\em could} be another group, but doesn't have to be).  Here are some relevant definitions.

\begin{itemize}

\item We will call the group $G$ and the set $X$.  An {\em action} of $G$ on $X$ is a function that sends pairs in $G \times X$ to things in $X$.  We write $(g,x) \mapsto gx$, satisfying
\begin{enumerate}
\item $ex = x$ for all $x \in X$
\item $(g_1g_2)x = g_1(g_2x)$ for all $x \in X$ and all $g_1, g_2\in G$.
\end{enumerate}
We call $X$ a {\em $G$-set}.

\item Two elements $x, y \in X$ are {\em $G$-equivalent} (written $x \sim y$) provided there is some $g \in G$ such that $gx = y$.

\item The {\em orbit} of an element $x \in X$ (written $\mathcal{O}_x$) is the set of all elements $y \in X$ that are $G$-equivalent to $x$.

\item The {\em fixed point set} of an element $g \in G$ (written $X_g$) is the set of all $x \in X$ such that $gx = x$.

\item The {\em stabilizer subgroup} of an element $x \in X$ (written $G_x$) is the set of all $g \in G$ such that $gx = x$.
\end{itemize}

To get used to these new definitions, let's work a few examples.

\begin{questions}
\question Let $X = \{1,2,3,4,5,6\}$ and $G = \{(1), (12)(3456), (35)(46), (12)(3654)\}$ (a subgroup of $S_6$).\\  $G$ acts on $X$ by $(\sigma,x) \mapsto \sigma(x)$. 
\begin{parts}
	\part For each $x \in X$, find $\mathcal{O}_x$, the orbit of $x$ in $G$.
	\vfill  
	\part For each $g \in G$, find $X_g$, the fixed point set of $g$ in $X$.
	\vfill
	\part For each $x \in X$, find $G_x$, the stabilizer subgroup of $x$ in $G$.
	\vfill
	\part Do you notice anything about the sizes of the sets you found, how many different sets you found, etc?
\end{parts}


\end{questions}



\end{document}



