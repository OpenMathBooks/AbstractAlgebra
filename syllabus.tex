\documentclass[12pt,letterpaper]{article}


\usepackage{fullpage, graphicx, url}
\usepackage{enumerate}
\usepackage{multicol}


\pagestyle{empty}
\thispagestyle{empty}


\begin{document}

\begin{center}\textbf{\Large Introduction to Abstract Algebra II}

\textbf{Math 322 Spring 2021 (3 credits)}
\end{center}
\vskip 2ex

\noindent\textbf{Professor}: Oscar Levin, Ph.D,~~ ~~ 970-576-0225 (cell),~~ \url{oscar.levin@unco.edu}

\noindent\textbf{Office}: \url{https://unco.zoom.us/my/oscarlevin} and Ross 2240D.

\noindent\textbf{Student Hours}: MWF 1:15-2:15.  Also other times by appointment.

\noindent\textbf{Textbook}: We will use a customized version of \emph{Abstract Algebra: Theory and Applications} by Tom Judson.  See Canvas for the link to the online textbook and notes.

\vskip 2 ex

Welcome to what promises to be yet another exciting and fun filled semester of Abstract Algebra!  I know you are all eager to get started, but please take a few moments to glance at this syllabus.
\vskip 2ex

\noindent
\textbf{Prerequisite}: MATH 321 with a grade of C or better
\vskip 2 ex

\noindent
\textbf{Course Description}: A continuation of MATH 321. Topics will include: groups, rings, fields and Galois theory, as well as applications of these topics to other areas in mathematics.

\vskip 2ex
\noindent
\textbf{Course Objectives}: Students successfully completing this class will have mastered a basic understanding of the concepts in abstract algebra, generally, and group, ring and field theory, specifically.  Additionally, students will improve their ability at reading, writing and validating proofs. 

\vskip 2 ex

\noindent
\textbf{Outline of Course Content}: Here is a list of topics grouped not chronologically, but topically.
\begin{enumerate}\itemsep1pt \parskip0pt \parsep0pt
\item Advanced group theory, the structure of finite groups, permutations.
\item Group actions and applications to combinatorics.
\item Further exploration of rings, polynomials, homomorphisms and quotient rings.
\item Field extensions, basic Galois theory, and applications to geometry and polynomials.
\end{enumerate}

\vskip 2ex

\noindent
\textbf{Method of Evaluation}: Your final letter grade will be calculated as follows:
\vskip 1ex

\begin{tabular}{llcll}
Homework: & 20\% & \qquad & Quizzes: & 10\% \\
Exams: & 20\% each & & Final Exam: & 15\% \\
Group project: & 10\% && Participation and Effort: & 5\%
\end{tabular}
\vskip 1ex
Grades will be assigned according to the following scale:
\begin{center}
\begin{tabular}{||c|c|c|c|c|c||}\hline
93-100\%: A& 90-92\%: A-  & 87-89\%: B+ & 83-86\%: B & 80-82\%: B- & 77-79\%: C+ \\
73-76\%: C & 70-72\%: C- & 67-69\%: D+ & 63-66\%: D & 60-62\%: D- & $\leq 59\%$: F \\ \hline
\end{tabular}
\end{center}

\vskip 2 ex

\noindent\textbf{Course Requirements}:
\vskip 1ex
\underline{Exams}: There will be two midterm exams and a cumulative final.  The midterm exams will contain both an in-class (timed) and take-home (untimed, open book) portion.  The take-home portion will be due the day of the in-class portion.  Midterms are \textit{tentatively} scheduled for the following dates:

\begin{tabular}{ll}
 Exam 1: & Monday, February 15 (take-home distributed Friday prior) \\

 Exam 2: & Monday, April 12 (take-home distributed Friday prior)
\end{tabular}

The cumulative final will a take-home exam, due \textbf{Thursday, May 6 at 1:30pm}.
\vskip 1 ex

\underline{Quizzes}: There will be two types of quizzes: frequent online reading quizzes (administered as \emph{Discussions} on Canvas) and occasional short (10 minute) take-home quizzes.  Quizzes will always be announced in Canvas; you should check for one for every class period (they will be due an hour before class).  Take-home quizzes will be very similar to homework, except that you will be expected to complete them on your own without referring to your notes. These quizzes will allow you to check yourself on some basic problems as we move through the semester, so that you will not be surprised when you get to the exams, and to ensure you are keeping up with the material. 

\vskip 1 ex
\underline{Homework}: Homework will be assigned daily.  There will be a mix of practice problems and problems to write up nicely.  
The practice problems do not need to be turned in, but you really need to do them so you are ready for class discussions or quizzes or exams.  For the problems I ask you to turn in, expect to submit one or two problems on Canvas each morning before class.  After you get feedback on these problems, you will be able to correct them and resubmit if you need to.

I strongly encourage you to type you homework, as this makes the correction and resubmission steps easier, as well as the online submission process.  I will provide you some suggestions of how to type mathematics effectively.  If you prefer to hand-write your homework, you will need to phone-scan your work to pdf to submit, and you will need to rewrite each problem neatly for the resubmission.

All homework will be listed on Canvas.

\vskip 1 ex

\underline{Group Project}: There are lots of interesting topics and applications of abstract algebra, and we will have time to study only a few of these.  The group project is an opportunity for groups of 2-3 students to research and present a topic of interest to them.  There will be both a class presentation (during our final exam period) and a short paper that your group will write.  I'll give you lots of help selecting and understanding the project.  More details will be provided soon.

\vskip 1 ex

\underline{Class-work}:  Class periods will be a mix of lecture, discussion, and group investigation, with an emphasis on the latter two.  Come to class ready to actively do some math.  Make sure that you have access to our OneNote class notebook, as that is where we will write notes and do group work.  

Outside of class, I encourage you to work in small groups as well.  Actively participating in your own learning, as well as helping your classmates, is the best way to succeed in the course.
\vskip 2 ex

\noindent
\textbf{Attendance Policy}: You are expected to attend every class period.  You are welcome to join the class in person (preferred when the option is available) or participate through zoom.  I will record our zoom sessions if you must miss, but note that the work you will do in breakout rooms is not recorded.  

For days that you attend class virtually, you should try to be connected to zoom for the entire 50 minutes of our class meetings, keep your camera on, and actively participate in group work breakout sessions during that time.

\vskip 2 ex

\noindent\textbf{Makeup Policy}: In general, missed exams may not be made up and homework may not be turned in late.  Exceptions will be made only in \emph{very} extreme cases.  Please contact me well in advance whenever possible if you need me to consider such an exception.  
\vskip 2ex


\noindent\textbf{In Person Classroom Policies}: Don't be rude.  Please be considerate of your fellow classmates and do not act in a disruptive manor.  Turn off your cell phones before coming to class and keep them put away, arrive on time, and do not pack up your things before the end of class.  You must wear a face covering while in the classroom, and observe social distancing requirements.  Due to the mask mandate, you should not bring food to class and limit the amount of water or other drink as much as possible.

\noindent\textbf{Zoom Classroom Policies}:  Whenever possible, you should have your video camera on, and microphone muted when you are not speaking.  During class time, you should be focused on the class; please do not try to multitask.  There will be many opportunities for you to speak in class, and you should never feel shy about unmuting yourself to speak.  If I feel like the same people are always talking and others are having trouble cutting in, I will be the one responsible for slowing us down and asking that you ``raise your hand.''  Until then interrupt as much as you like.

\vskip 2 ex

\noindent\textbf{Statement of Academic Integrity}: Don't cheat!  It is expected that members of this class will observe strict policies of academic honesty.  In particular, you are expected to solve homework problems by yourself or together with your group, and not find solutions online.  In general, UNC's policies and recommendations for academic misconduct will be followed. For additional information, please see the Student Code of Conduct at the Dean of Student's website \url{http://www.unco.edu/dos/Conduct/codeofconduct.html}. In the case of academic appeals, university procedures will be followed. For information on academic appeals, see \url{http://www.unco.edu/regrec/Current%20Students/AcademicAppeals.html}.

\vskip 2 ex

\noindent\textbf{Disability Resources}: It is the policy and practice of the University of Northern Colorado to create inclusive learning environments.  If there are aspects of the instruction or design of this course that present barriers to your inclusion or to an accurate assessment of your achievement (e.g. time-limited exams, inaccessible web content, use of videos without captions), please communicate this with your professor and contact Disability Resource Center (DRC) to request accommodations.  Office: (970) 351-2289, Michener Library L-80. Students can learn more about the accommodation process at  \url{https://www.unco.edu/disability-resource-center/}.

\vskip 2ex
\noindent\textbf{Suggestions for a Successful Semester}:
\begin{enumerate}\itemsep1pt \parskip0pt \parsep0pt
\item Your \textbf{JOB} as a student of mathematics is to \textbf{ask questions}.  This can be difficult but it is an important skill that will serve you well.  Use this class as a safe place to practice.  My promise: \underline{any} question you ask will only ever \underline{improve} my opinion of you.
\item Think critically! Don't believe something just because I tell you that it's true. Always
ask yourself if you have good reason to believe it.
\item Do all the practice and assigned homework, as soon as possible.  Practice, practice, etc.
\item Challenge yourself.  Some topics we study might come easy to you, others not.  You should look for these challenges, work hard, and overcome them.  You are here to learn, not to demonstrate what you already know.
\setcounter{enumi}{5}
\item Don't skip numbers.
\item Work with others.  We will do a lot of group work in class.  There is no reason you can't continue to work with your new friends on the homework and when studying for exams.  Teaching each other mathematics is the best way to learn it.
\item Don't ride a bike without a helmet.  Especially a stationary bike!
\item If you need help, come see me in my office during the hours listed above or make an appointment with me for some other time. My door is always figuratively open when it's literally open.
\item Most importantly, if you don't understand something: ASK!  See suggestion number 1.  You are one of only a handful of students in the class, so please please please interrupt when something is unclear.
\end{enumerate}




\end{document}
