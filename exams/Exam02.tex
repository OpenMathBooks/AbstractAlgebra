\documentclass[11pt]{exam}

\usepackage{amssymb, amsmath, amsthm, mathrsfs, multicol, graphicx} 
\usepackage{tikz}

\def\d{\displaystyle}
\def\?{\reflectbox{?}}
\def\inv{^{-1}}
\def\b#1{\mathbf{#1}}
\def\f#1{\mathfrak #1}
\def\c#1{\mathcal #1}
\def\s#1{\mathscr #1}
\def\r#1{\mathrm{#1}}
\def\N{\mathbb N}
\def\Z{\mathbb Z}
\def\Q{\mathbb Q}
\def\R{\mathbb R}
\def\C{\mathbb C}
\def\F{\mathbb F}
\def\A{\mathbb A}
\def\X{\mathbb X}
\def\E{\mathbb E}
\def\O{\mathbb O}
\def\FR{\mathscr{F(\R)}}
\def\pow{\mathscr P}
\def\inv{^{-1}}
\def\nrml{\triangleleft}
\def\st{:}
\def\~{\widetilde}
\def\rem{\mathcal R}
\def\iff{\leftrightarrow}
\def\Iff{\Leftrightarrow}
\def\and{\wedge}
\def\And{\bigwedge}
\def\AAnd{\d\bigwedge\mkern-18mu\bigwedge}
\def\Vee{\bigvee}
\def\VVee{\d\Vee\mkern-18mu\Vee}
\def\imp{\rightarrow}
\def\Imp{\Rightarrow}
\def\Fi{\Leftarrow}

\def\={\equiv}
\def\var{\mbox{var}}
\def\mod{\mbox{Mod}}
\def\Th{\mbox{Th}}
\def\sat{\mbox{Sat}}
\def\con{\mbox{Con}}
\def\bmodels{=\joinrel\mathrel|}
\def\iffmodels{\bmodels\models}
\def\dbland{\bigwedge \!\!\bigwedge}
\def\dom{\mbox{dom}}
\def\rng{\mbox{range}}
\DeclareMathOperator{\wgt}{wgt}
\DeclareMathOperator{\Gal}{Gal}
\DeclareMathOperator{\ord}{ord}


\def\bar{\overline}

%\pointname{pts}
\pointsinmargin
\marginpointname{pts}
\marginbonuspointname{bn-pts}

\addpoints
\pagestyle{headandfoot}
% \printanswers

\firstpageheader{Math 322}{\bf\large Exam 2}{Spring 2020}
\runningfooter{}{\thepage}{}
\extrafootheight{-.45in}



\begin{document}
\noindent{\bf Instructions:} Write all answers in the space provided, showing work and providing explanations.  Open notes, open book, but you must WORK ALONE and may NOT search the internet for answers. 

\begin{center}
	\emph{By signing below, I certify that the work on this take-home exam is solely my own, that I did not receive assistance from anyone other than my instructor, and did not use resources other than my own notes and the course textbook.}

	\vskip 1em
	\noindent {\bf Signature:} \underline{\hspace{3in}} \hfill {\bf Date:} \underline{\hspace{1.5in}}
\end{center}
\hrulefill

\begin{questions}



\question[8] Consider the permutation $(13)(14)(23)$.  Rewrite this permutation as a product of three transpositions where 3 \underline{only} occurs in the first transposition (reading from the left).  Explain how you know what you found is equal to the original permutation.
\begin{solution}
 $(13)(14)(23) = (13)(23)(14) = (23)(12)(14)$.  We can find this by noting that $(14)(23)$ are disjoint, so they commute.  Then $(13)(23) = (13)(32) = (132)$.  But $(132) = (321) = (32)(21)$ which can be written as $(23)(12)$.  
 
 Alternatively, to see that $(13)(14)(23) = (23)(12)(14)$ just write both as products of disjoint cycles, or in this case, a single cycle:
 
 \[(13)(14)(23) = (1432) \qquad (23)(12)(14) = (1432)\]
\end{solution}
\vfill
\question[6] Is it possible to rewrite $(13)(14)(23)$ as the product of 6 transpositions?  Of 9 transpositions? Explain.
\vfill

\clearpage

\question[12] Let $G$ be a group containing elements $\{e, a, b, c, d, f, g, h\}$.  Here is part of the table for $G$:
\begin{center}
  \begin{tabular}{c|cccccccc}
         & $e$ & $a$ & $b$ & $c$ & $d$ & $f$ & $g$ & $h$ \\ \hline
     $\vdots$ & &&& \vdots & & & & \\
     $c$ & $c$ & $b$ & $e$ & $a$ & $h$ & $g$ & $d$ & $f$ \\ 
     $\vdots$ & &&& \vdots & & & & 
   \end{tabular}
\end{center}
\begin{parts}
  \part Find the left regular representation of $c$ as in Cayley's theorem.  That is, find the permutation $\lambda_c$ in $S_8$ corresponding to $c$.
  \begin{solution}
   $\lambda_c = \begin{pmatrix} 1 & 2 & 3 & 4 & 5 & 6 & 7 & 8 \\ 4 & 3 & 1 & 2 & 8 & 7 & 5 & 6\end{pmatrix} = (1423)(5867)$
  \end{solution}

  \vfill
  \part Suppose $\lambda_d = (1526)(37)(48)$ is the left regular representation of $d$.  Write the row for $d$ in the group table. 
  \begin{solution}
   The row will read $d~f~g~h~a~e~b~c$
  \end{solution}

  \vfill
  \part If you compute $\lambda_c\lambda_d$ you get another permutation in $S_8$.  Of which element in $G$ is it the left regular representation?  That is, find $x \in G$ such that $\lambda_x = \lambda_c\lambda_d$.  Briefly explain how you know you are right.
  \begin{solution}
   $\lambda_c\lambda_d = \lambda_{cd}$ by the way we define regular representations.  But because $G$ is isomorphic to the group of regular representations, we know $\lambda_{cd}$ corresponds to $cd$ which in $G$ equals $h$.  Thus $\lambda_c\lambda_d = \lambda_h$.
  \end{solution}

  \vfill
  \vfill
\end{parts}

\clearpage

% \question[16] Consider groups generated by two elements $G = \langle a, b \rangle$.  Note that these groups \emph{could} still be cyclic, depending on the relationship between $a$ and $b$.
% \begin{parts}
% \part Suppose $G = \langle a, b \rangle$ is abelian with $\ord(a) = 4$ and $\ord(b) = 5$.  List all the elements of $G$.  What familiar group is $G$ isomorphic to?  Justify your answer using internal direct products.  Hint: one of the elements you list should be either $a^2b^3$ or $b^3a^2$ (these are the same element).

% \begin{solution}
% We have $\{e, a, a^2, a^3, b, ba, ba^2, ba^3, b^2, b^2a, b^2a^2, b^2a^3, b^3, b^3a, b^3a^2, b^3a^3, b^4, b^4a, b^4a^2, b^4a^3\}$.  This is everything since $a$ and $b$ commute.  We can view $G$ as the internal direct product of $\langle a \rangle$ and $\langle b \rangle$.  Since $\langle a \rangle \cong \Z_4$ and $\langle b \rangle \cong \Z_5$, we see that $G \cong \Z_4\times \Z_5 \cong \Z_{20}$ (since 4 and 5 are relatively prime).  

% Note that in this example, $G$ is actually cyclic, generated by $ab$.
% \end{solution}

% \part Find an example of an abelian group $G = \langle a, b\rangle$ where the order of $G$ is strictly less than $\ord(a)\cdot \ord(b)$.  (Do not assume the orders are the same as the previous part.) Explain why your example works.

% \begin{solution}
% In general, if $b$ is a power of $a$, then adding $b$ to the list of generators will not change the order, and we will get a group of size $\ord(a)$.  For example, consider $a = 1$ and $b = 3$ under the operation of addition mod 6.  This gives $\Z_6$ even though $\ord(1) = 6$ and $\ord(3) = 2$.
% \end{solution}

% \part Prove that if $G = \langle a, b \rangle$ is abelian, then $|G| \le \ord(a)\cdot \ord(b)$.

% \begin{solution}
% Every element will be of the form $a^jb^k$: we can always rearrange any product of $a$'s and $b$'s so all the $a$'s come first.  This is where we use the fact that $G$ is abelian.  There are exactly $\ord(a)$ different choices for $j$ and exactly $\ord(b)$ choices for $k$, giving $\ord(a)\ord(b)$ choices for elements of this form.  If all of these are distinct, then $|G| = \ord(a)\ord(b)$, but of course some of these might be repeats, as seen in the example above.
% \end{solution}

% \part Give an example of a non-abelian group $G = \langle a,b\rangle$ that shows the previous part is not true for groups in general. Explain why your example works and what goes wrong when you try to use the proof you gave in the previous part.
% \begin{solution}
% For example, $S_5 = \langle (12), (12345)\rangle$.  We have $\ord((12)) = 2$ and $\ord((12345)) = 5$ but $|S_5| = 120 \not\le 10$.  The problem is that here $(12)(12345)(12) \ne (12345)$ for example.  Since we cannot rearrange the generators to group them together, we get more elements.  In fact, it is possible to get an infinite group using two generators both with finite order.
% \end{solution}


% \end{parts}




\question[24] Consider the group $\Z_{75}$.
\begin{parts}
\part Give an example of a subnormal series which is not a composition series.  Explain why your example works.
\begin{solution}
$\Z_{75} \supset \langle 25\rangle \{0\}$ is such a subnormal series.  It is subnormal since every subgroup is normal in its predecessor ($\Z_{75}$ is abelian so all subgroups are normal).  However, $\Z_{75}/\langle 25\rangle \cong \Z_{25}$ which is not simple.
\end{solution}

\vfill
\part Find a \emph{refinement} of the series you gave above which is a composition series.  That is, show how to extend your series into a composition series.

\begin{solution}
We can add a subgroup between $\Z_{75}$ and $\langle 25\rangle$ since that quotient group is not simple.  We get
\[\Z_{75} \supset \langle 5 \rangle \supset \langle 25 \rangle \supset \{0\}\]
This is a composition series now since the quotient groups, reading from left to right are $\Z_5$, $\Z_5$ and $\Z_3$, all simple groups.
\end{solution}
\vfill
\part Find \emph{all} other composition series and briefly explain how you know you have them all.

\begin{solution}
In addition to the one found above, we have
\[\Z_{75} \supset \langle 3 \rangle \supset \langle 25 \rangle \supset \{0\}\]
\[\Z_{45} \supset \langle 5 \rangle \supset \langle 15 \rangle \supset \{0\}\]
This is all of them.  By the Jordan-H\"older theorem, all composition series must be isomorphic, which means they have the same quotient groups just possibly in a different order.  The quotient groups must be $\Z_3$, $\Z_5$ and $\Z_5$, and these can come in one of three orders (based on where the $\Z_3$ falls).  Once we have the quotient groups, there is only one choice for subgroup (in this case) that gives that quotient group.
\end{solution}
\vfill
\part If $p(x)$ is a polynomial whose splitting field has Galois group over $\Q$ isomorphic to $\Z_{75}$, will the roots of $p(x)$ be expressible in terms of $n$th roots and field operations?  Briefly explain (you can cite a result we discussed in class).

\begin{solution}
Yes.  Since the Galois group for $p(x)$ is solvable, we know the polynomial will be solvable by radicals.
\end{solution}
\vfill
\end{parts}

\clearpage
\question[12] Suppose $E$ is a splitting field whose Galois group over $\Q$ is isomorphic to $\Z_{75}$.  (For this problem, you can refer to the work you did on the previous problem, part (c) in particular, as well as the Fundamental Theorem of Galois Theory.).

\begin{parts}
  \part How many different intermediate fields are there between $\Q$ and $E$?  What degree extensions are these?  Justify your answers.  You may refer to the work you did on the previous problem (part (c) in particular) 
  \vfill
  \part How many of the intermediate fields between $\Q$ and $E$ are splitting fields for some polynomial?  Briefly explain.
  \vfill
\end{parts}



\clearpage


% \question[16] For each of the statements below, decide whether they are \textbf{TRUE} or \textbf{FALSE}.  Then justify your choices either with a brief explanation (if true) or counterexample (if false).
% \begin{parts}
% \part Every element in $S_8$ can be written as the product of 8 transpositions.
% \begin{solution}
% False.  For example, the element $(1234) = (12)(23)(34)$ can be written as an odd number of permutations, so every way to write it will use an odd number of permutations.
% \end{solution}
% \vfill
% \part Every element of $A_8$ can be written as the product of 8 transpositions.
% \begin{solution}
% True.  $A_8$ contains all the even permutations in $S_8$.  Even permutations like $(12)(23)$ can still be written as 8 transpositions by adding on $(12)(12)$ enough times to bring the total number up to 8.  We must simply check that no element in $A_8$ requires an even number of transpositions \emph{larger} than $8$.  Every element in $A_8$ can be written as the product of disjoint cycles.  These could be a single cycle of length 7 (which gives exactly 8 transpositions) or less, as a 2-cycle and a 6 (or smaller) cycle (giving 8 transpositions), a 3-cycle and a 5-cycle (giving 8 transpositions), etc.
% \end{solution}
% \vfill
% \part For any group $G$, all subnormal series of $G$ are the same length.
% \begin{solution}
% False.  For example, $\Z_{12} \supset \langle 6\rangle \supset \{0\}$ and $\Z_{12} \supset \langle 2\rangle \supset \langle 6 \rangle \supset \{0\}$ are both subnormal series, but have different length.  
% However, all \emph{composition} series have the same length by the Jordan-H\"older theorem.
% \end{solution}
% \vfill
% % \part For any group $G$, all composition series of $G$ are the same length.
% % \begin{solution}
% %   True.  This is implied by the Jordan-H\"older theorem.
% % \end{solution}
% % \part For any group $G$, if $m$ divides the order of $G$, then there is an element of $G$ with order $m$.
% % \begin{solution}
% % False.  This would be true if $m$ were prime (by Cauchy's theorem) but in general this is false.  For example, $S_5$ has order 120, but there is no element with order larger than 6.  Note that if the statement were true, then all groups would be cyclic.
% % \end{solution}
% \part If $a$ has order 5 then the cyclic group $\langle a\rangle$ is isomorphic to a subgroup of $S_5$.
% \begin{solution}
%   True, by Cayley's theorem, since $\langle a \rangle$ has 5 elements.
% \end{solution}
% \vfill

% \end{parts}

\question[12] Consider the group $U(14297)$.  In completing this question, you should NOT try to list out the elements of the group. 
\begin{parts}
  \part What is the order of the group (i.e., how many elements are in the group)?  Explain how you know.  Hint: $14297 = 17\cdot 29^2$
  \vfill
  \part Find $8640^{12992}$ modulo 14297, and explain how you know your answer is correct without using a calculator (and how this relates to the previous part).
  \vfill
  \part Let $E = 321$.  Find a number $D$ such that $8640^{ED} \equiv 8640 \pmod{14297}$.  How do you know you are correct?  If you use technology to find $D$, explain what the technology is giving you.
  \vfill
\end{parts}

\clearpage
% %change numbers?
% \question[6] Find all abelian groups of order $1125 = 3^2 \cdot 5^3$ which contain at least one element of order 25.  Briefly explain how you know you have them all.

% \begin{solution}
% By the Fundamental Theorem of Finite Abelian Groups, we know that these groups can be written as direct product of cyclic $p$-groups.  Each abelian group will be isomorphic to one of the following:
% \[\Z_{3^2} \times \Z_{5^3} \qquad \Z_{3^2} \times \Z_{5^2}\times \Z_5\qquad \Z_{3^2} \times \Z_{5}\times \Z_5\times \Z_5\]
% \[\Z_{3}\times \Z_3 \times \Z_{5^3} \qquad \Z_{3}\times \Z_3 \times \Z_{5^2}\times \Z_5\qquad \Z_{3}\times \Z_3 \times \Z_{5}\times \Z_5\times \Z_5\]

% Only 4 of these have elements of order 25:

% \[\Z_{3^2} \times \Z_{5^3} \qquad \Z_{3^2} \times \Z_{5^2}\times \Z_5 \qquad \Z_{3}\times \Z_3 \times \Z_{5^3} \qquad \Z_{3}\times \Z_3 \times \Z_{5^2}\times \Z_5 \]

% The two we excluded do not have elements of order 25 since when constructing the direct products we considered the $5$-groups.  If we had an element of order 25, we would have taken the subgroup generated by such an element and that would have given us $\Z_{25}$ as part of the direct product.

% Note that $\Z_{125}$ does have an element of order 25, namely $5 \in \Z_{125}$.
% \end{solution}

% \vfill
% \question[4] Of the groups (of order 1125) described in the previous question, how many are cyclic?  Explain.
% \begin{solution}
% Only one: $\Z_{9}\times \Z_{125}$.  Since $9$ and $125$ are relatively prime, this is isomorphic to $\Z_{1125}$, which is cyclic (alternatively, the element $(1,1)$ generates).  Note that all the other groups still have order $1125$ but are not isomorphic to this one (they are different!) so they cannot be cyclic (there is only one cyclic group of each order).
% \end{solution}

\question[8] For how many abelian groups $G$ of order less than 100, is $G = G(7)G(11)$?  (Recall that $G(p)$ is the set of all elements in $G$ of order $p^k$ for some $k\ge 0$.)  List all such groups and explain your answer.
\begin{solution}
  We know that $G(7)$ will be isomorphic to either $\Z_{1}$, $\Z_7$, $\Z_{49}$, or $\Z_{7}\times \Z_{7}$ (any larger group will have more than 100 elements).  Similarly, $G(11)$ is isomorphic to either $\Z_1$, $\Z_{11}$.

  Further, we know that the inner direct product $G(7)G(11)$ will be isomorphic to $G(7)\times G(11)$.  Thus our possible groups are
  \[\Z_1, \qquad \Z_{7}, \qquad \Z_7\times \Z_7, \qquad \Z_{49}\]
  \[\Z_{11}, \qquad \Z_{7}\times \Z_{11}.\]
\end{solution}


\vfill


\question[6] Let $G$ be a group of order $|G| = 16$, that acts on the set $X = \{1,2,3,4,5\}$.  Prove that there is at least one element $x \in X$ such that $gx = x$ for all $g \in G$.  

Hint 1: What do you need to show about $\mathcal O_x$?   

Hint 2: What sizes could the stabilizer subgroups of $G$ have?  

\begin{solution}
  We need to show that for some $x$, $\mathcal O_x = \{x\}$.  The orbit-stabilizer theorem says that $|G| = |\mathcal O_x|\cdot |G_x|$.  This says in particular that $|\mathcal O_x|$ must divide the size of $G$.  So the only options for $|\mathcal O_x|$ are 1, 2, or 4.  But the orbits partition $X$, so it cannot be that they are all 2 and 4.
\end{solution}

\vfill

\clearpage

\question[12] You will create a cube by connecting 12 pipe cleaners  of equal length together (to form the edges of the cube).  The pipe cleaners come in 3 different colors (and you have at least 12 of each color you could use).  How many different cubes could you make?  Show your work.

\vfill
\vfill

\bonusquestion[10] BONUS! Suppose you made a cube as in the previous problem but required that you used an equal number of each color (so four of each).  Now how many different cubes could you make?
\vfill

\end{questions}




\end{document}


