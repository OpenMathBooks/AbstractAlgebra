\documentclass[10pt]{exam}

\usepackage{amssymb, amsmath, amsthm, mathrsfs, multicol, graphicx} 
\usepackage{tikz}

\def\d{\displaystyle}
\def\?{\reflectbox{?}}
\def\inv{^{-1}}
\def\b#1{\mathbf{#1}}
\def\f#1{\mathfrak #1}
\def\c#1{\mathcal #1}
\def\s#1{\mathscr #1}
\def\r#1{\mathrm{#1}}
\def\N{\mathbb N}
\def\Z{\mathbb Z}
\def\Q{\mathbb Q}
\def\R{\mathbb R}
\def\C{\mathbb C}
\def\F{\mathbb F}
\def\A{\mathbb A}
\def\X{\mathbb X}
\def\E{\mathbb E}
\def\O{\mathbb O}
\def\FR{\mathscr{F(\R)}}
\def\pow{\mathscr P}
\def\inv{^{-1}}
\def\nrml{\triangleleft}
\def\st{:}
\def\~{\widetilde}
\def\rem{\mathcal R}
\def\iff{\leftrightarrow}
\def\Iff{\Leftrightarrow}
\def\and{\wedge}
\def\And{\bigwedge}
\def\AAnd{\d\bigwedge\mkern-18mu\bigwedge}
\def\Vee{\bigvee}
\def\VVee{\d\Vee\mkern-18mu\Vee}
\def\imp{\rightarrow}
\def\Imp{\Rightarrow}
\def\Fi{\Leftarrow}

\def\={\equiv}
\def\var{\mbox{var}}
\def\mod{\mbox{Mod}}
\def\Th{\mbox{Th}}
\def\sat{\mbox{Sat}}
\def\con{\mbox{Con}}
\def\bmodels{=\joinrel\mathrel|}
\def\iffmodels{\bmodels\models}
\def\dbland{\bigwedge \!\!\bigwedge}
\def\dom{\mbox{dom}}
\def\rng{\mbox{range}}
\DeclareMathOperator{\wgt}{wgt}
\DeclareMathOperator{\Gal}{Gal}


\def\bar{\overline}

%\pointname{pts}
\pointsinmargin
\marginpointname{pts}
\marginbonuspointname{bn-pts}

\addpoints
\pagestyle{headandfoot}
%\printanswers

\firstpageheader{MATH 322}{\bf\large Exam 1}{Spring 2021}
\runningfooter{}{\thepage}{}
\extrafootheight{-.45in}



\begin{document}

\noindent{\bf Instructions:} Write all answers in the space provided, showing work and providing explanations.  Open notes, open book, but you must WORK ALONE and may NOT search the internet for answers.  

\begin{center}
	\emph{By signing below, I certify that the work on this take-home exam is solely my own, that I did not receive assistance from anyone other than my instructor, and did not use resources other than my own notes and the course textbook.}

	\vskip 1em
	\noindent {\bf Signature:} \underline{\hspace{3in}} \hfill {\bf Date:} \underline{\hspace{1.5in}}
\end{center}

\hrulefill


\begin{questions}

	\question[15] Give an example of each of the following.  For each, explain how you know your example is correct (that is, say why your example really is an example of what was requested). 
	\begin{parts}
		\part An algebraic number that is not constructible using a straight-edge and compass.
		\begin{solution}
			For example, $\sqrt[3]{2}$ is algebraic (since it is the root of the polynomial $x^3 - 2$) but not constructible: it's minimal polynomial has degree 3, so there is no way that $\sqrt[3]{2}$ can be in a field extension of degree $2^k$ for some $k$.
		\end{solution}
		\vfill

		\part An extension field $E$ of $\Q$ such that every element of $E$ is the root of a degree 4 polynomial.
		\begin{solution}
			We could take $E = \Q(\sqrt[4]{2})$.  This is a degree 4 extension, meaning every element is the root of a polynomial of degree at least 4.  For any elements that are roots of degree 1 or 2 polynomials, we could simply multiply by $x^3$ or $x^2$ to get a polynomial of degree 4 that still has the number as a root.
		\end{solution}

		\vfill

		\part Two extension fields $E$ and $F$ of $\Q$ such that $[E:\Q] = [F:\Q]$ but $E$ is not isomorphic to $F$.  (Make sure to explain why your example satisfies both properties.)
		\begin{solution}
			For example, $E = \Q(\sqrt{2})$ and $F = \Q(\sqrt{3})$.  These both have degree 2 (looking at the minimal polynomials for $\sqrt{2}$ and $\sqrt{3}$), but since those numbers are not roots of the same irreducible polynomial, there is no isomorphism between the fields.
		\end{solution}
		\vfill
	\end{parts}

\clearpage
\question[16] Let's look at the field $E = \Q(\sqrt{5}, \sqrt[3]{5})$ and its subfields.  Find the degree of the field extensions requested below and explain how you know you are correct.
	
	\begin{parts}
	\part $\,[\Q(\sqrt{5}):\Q]$.
	\begin{solution}
	The degree is 2, since $\sqrt{5}$ has minimum polynomial $x^2 - 5$ (irreducible by Eisenstein's criterion).
	\end{solution}
	\vfill
	\part $\,[\Q(\sqrt[3]{5}):\Q]$.
	\begin{solution}
	The degree is 3, as $\sqrt[3]{5}$ is the root of the irreducible polynomial $x^3 - 5$.
	\end{solution}
	\vfill
	\part $\,[E:\Q]$.
	\begin{solution}
	Since $\Q(\sqrt{5})$ and $\Q(\sqrt[3]{5})$ are both subfields of $E$, the degree of $E$ must be a multiple of both $2$ and $3$, so at a minimum, must be 6.  But $E \subseteq \Q(\sqrt{5}, \sqrt[3]{5})$ which definitely has degree 6, so the degree of $E$ over $\Q$ is indeed 6.
	\end{solution}
	\vfill
	\part $\,[E:\Q(\sqrt{5})]$.
	\begin{solution}
	The degree must be 3, since the degree of $E$ over $\Q$ will be 6 (see above).  We cannot have the degree be 1, because that would mean that $\sqrt[3]{5}$ was already in $\Q(\sqrt{5})$, but that would mean the degree of $\Q(\sqrt[3]{5})$ over $\Q$ would be at most 2 (it is not).  The degree cannot be 2 either, because that would make $[E:\Q]$ at most 4, but it needs to be a multiple of 3 since $E$ is an extension of $\Q(\sqrt[3]{5})$.
	\end{solution}
	\vfill
	\end{parts}
	
	\question[6] Use the previous problem to prove that there do not exists any rational numbers $a$, $b$, and $c$ such that $\sqrt{5} = a + b\sqrt[3]{5} + c \sqrt[3]{5}^2$.  Make sure to say what this has to do with fields.
	\begin{solution}
		This is saying that $\sqrt{5}$ is not an element of $\Q(\sqrt[3]{5})$, and we can verify that is true using the parts above.  If it was in that field, the the top field would be degree 3, not 6.
	\end{solution}
	\vfill
	\vfill
	\clearpage
% \question[8] Suppose $a(x)$ is a polynomial of degree 5 in $\Q[x]$ that has $\sqrt[3]{6}$ as a root.  Prove that $a(x)$ is NOT irreducible.  Your proof should use ideals and facts about ideals in $\Q[x]$.

% \begin{solution}
% Consider the ideal $J$ of all polynomials that have $\sqrt[3]{6}$ as a root.  This really is an ideal (for example, it is the kernel of the evaluation homomorphism $\sigma_{\sqrt[3]{6}}$ which evaluates a given polynomial at $\sqrt[3]{6}$).

% The ideal is an ideal of the ring $F[x]$, so is principle.  In other words, $J = \langle p(x)\rangle$ for some polynomial $p(x)$.  But $p(x)$ must be equal to $x^3 - 6$ as this polynomial is irreducible an has $\sqrt[3]{6}$ as a root (if it wasn't, then $x^3 - 6$ would be a multiple of $p(x)$, which is impossible since $x^3 - 6$ is irreducible).  

% Thus every polynomial in $J$ is a multiple of $x^3 - 6$, so in particular there cannot be any degree 4 or greater polynomial in $J$ which is irreducible.  Thus $a(x)$ is not irreducible.
% \end{solution}
% \vfill
\clearpage


\question[24] The polynomial $p(x) = x^4 - 10x^2 + 25x - 5$ is irreducible. Say $\varrho$ is one of the roots of $p(x)$.  While $\varrho$ is not an element of $\Q$, we can form a smallest possible extension field of $\Q$ that contains $\varrho$.  There are two ways to represent $E$: one as a quotient ring, the other as $\Q(\varrho)$.  
\begin{parts}

\part Carefully describe these two representations: the general form of elements in each.  Also, provide at least two specific examples of elements in both representations and how they are related. 
\begin{solution}
The representation $\Q(\varrho)$ is the set of all elements of the form $a + b\varrho + c\varrho^2 + d\varrho^3$ where $a, b, c, d \in \Q$.  We know that we don't need a $\varrho^4$ term because the degree of the field extension over $\Q$ is 4, as $\varrho$ is the root to an irreducible degree 4 polynomial.

The other representation is the quotient ring $\Q[x]/\langle p(x) \rangle$, in which elements are cosets $\langle p(x) \rangle + a(x)$ for all polynomials $a(x) \in \Q[x]$.  But by the division algorithm we can always pull out multiples of $p(x)$ for any $a(x)$ with degree greater than 3, so we can assume $a(x)$ has degree at most 3.

These two representations are isomorphic by the evaluation map that evaluates the polynomial $a(x)$ at $\varrho$.  For example, in $\Q(\varrho)$ the element $3 + 2\varrho - \varrho^2 + 7 \varrho^3$ corresponds to the coset $\langle p(x) \rangle + 3 + 2x - x^2 + 7x^3$.  Or going the other way, the coset $\langle p(x) \rangle + 5+ 4x^3$ corresponds to the element $5 + 4\varrho^4$.
\end{solution}
\vfill
\part Write $\varrho^5 - 7\varrho^3 + 1$ in terms of the basis for $\Q(\varrho)$ (first find $\varrho^4$ and $\varrho^5$).  Then use long division to find the remainder when $x^5 - 7x^3 + 1$ is divided by $p(x)$.  What do you notice and why does it make sense?

%Another question along these lines: Prove the remainder theorem.  Then explain why the remainder theorem is true using quotient rings.
\begin{solution}
From the fact that $\varrho$ is a root of $p(x)$ we have that $\varrho^4 = 10\varrho^2 - 25\varrho+5$.  Of course $\varrho^5 = \varrho\cdot \varrho^4$.  Thus the element we have here is \[\varrho(10\varrho^2 - 25\varrho+5) - 7 \varrho^3 + 1 = 3\varrho^3 - 25 \varrho^2 + 5\varrho + 1\]
In the quotient ring this is
\[\langle p(x) \rangle + 3x^3 - 25x^2 + 5x + 1\]

What does this have to do with divisions and remainders?  Well $x^5 - 7x^3 + 1 \in \langle p(x) \rangle + 3x^3 - 25x^2 + 5x + 1$ is saying that after dividing $x^5 - 7x^3 + 1$ by $p(x)$ you are left with a remainder of $3x^3 - 25x^2 + 5x + 1$.  But we could get this remainder not be actually dividing, but by plugging in $10x^2 - 25x + 5$ in for $x^4$.
\end{solution}
\vfill
\clearpage
\uplevel{Continuing from the previous page...}
\part The field $E$ is really a field, so there is a multiplicative inverse for every non-zero element.  Find the multiplicative inverse of $\varrho^3 - 4\varrho^2 + 6\varrho + 1$ using polynomials.  Show all your work.

\begin{solution}
We will find the inverse of the coset $\langle p(x)\rangle + x^3 - 4x^2 + 6x + 1$ in $\Q[x]/\langle p(x)\rangle$.  Using long division, we get $p(x) = (x+4)(x^3 - 4x^2 + 6x + 1) - 9$.  Thus
\[1 = \frac{-1}{9} p(x) + \left(\frac{1}{9}x + \frac{4}{9}\right)(x^3 - 4x^2 + 6x + 1).\]
In other words, $1 \in \langle p(x) \rangle + \left(\frac{1}{9}x + \frac{4}{9}\right)(x^3 - 4x^2 + 6x + 1)$ so $(\langle p(x)\rangle + \left(\frac{1}{9}x +\frac{4}{9}\right))(\langle p(x)\rangle + (x^3 - 4x^2 + 6x + 1) = \langle p(x) \rangle + 1$.  

Transferring back to $E$, we find that the inverse of $\varrho^3 - 4\varrho^2 + 6\varrho + 1$ is $\frac{1}{9}\varrho + \frac{4}{9}$.

\end{solution}
\vfill
\part Let $K$ be the splitting field for $p(x)$ (the specific polynomial from the previous page).  What are the three possible values for $[K:\Q]$?  In particular, explain why $[K:\Q]$ cannot be 4.  (It might be helpful to graph $p(x)$.)
	\begin{solution}
	 If you graph $p(x)$ you see that there are two real roots, so there must be 2 complex roots as well.  If $\varrho$ is a real root, there is no way to get the complex root from this in $\Q(\varrho)$.  Of course we don't know which root $\varrho$ is, but it doesn't matter: if $\varrho'$ is a different root of $p(x)$ then $\Q(\varrho) \cong \Q(\varrho')$ since they are both isomorphic to the same quotient ring.	

	 We know that $[E:\Q] = 4$ since $p(x)$ is irreducible of degree 4.  But since there are roots of $p(x)$ that are not in $E$, we know that $K$ is a strictly larger field.  We could have $[K:\Q]$ be $8$, $12$, or $24$ depending on how $p(x)$ factors in $E$.
	\end{solution}
\end{parts}
\vfill

\clearpage
\question[24] As we saw in class, the polynomial $p(x) = x^3 - 2$.  has three roots: $\sqrt[3]{2}$, $\sqrt[3]{2}\omega$, and $\sqrt[3]{2}\omega^2$ where $\omega = e^{i2\pi/3}$.  The splitting field $E$ has Galois group $G(E/\Q)$ isomorphic to $S_3$.  (Note, $\omega$ is a root of $x^2 + x + 1$ since $x^3 - 1 = (x-1)(x^2+x+1)$)
\begin{parts}
	\part We can write $E = \Q(\sqrt[3]{2}, \sqrt[3]{2}\omega)$ or $E = \Q(\sqrt[3]{2}, \omega)$.  Explain how we know these are the same.  Then write down two bases for $E$ using these two representations.
	\begin{solution}
		The fields are the same because we can get $\omega$ by dividing $\sqrt[3]{2}\omega$ by $\sqrt[3]{2}$, and obviously can get $\sqrt[3]{2}\omega$ as a product of $\sqrt[3]{2}$ and $\omega$.  
		
		One basis is $\{1, \sqrt[3]{2}, \sqrt[3]{2}^2, \sqrt[3]{2}\omega, \sqrt[3]{2}^2\omega, 2\omega\}$.  Another is $\{1, \sqrt[3]{2}, \sqrt[3]{2}^2, \omega, \sqrt[3]{2}\omega, \sqrt[3]{2}^2 \omega\}$.  
	\end{solution}
	\vfill
\part One element of the Galois group can be described as \[\sigma = \begin{pmatrix}\sqrt[3]{2} & \omega \\ \sqrt[3]{2}\omega & \omega^2\end{pmatrix}.\]
	Write the five other elements of the Galois group in this form (this is different from the way we wrote them in class, but that is the point of this problem).
	\begin{solution}		
		If we send $\sqrt[3]{2}$ to $\sqrt[3]{2}\omega$ and $\omega$ to its conjugate, we would send $\sqrt[3]{2} + \omega$ to $\sqrt[3]{2}\omega + \omega^2$.  Since $\omega^2 = -\omega - 1$ this becomes $\sqrt[3]{2} \omega - \omega - 1$.
	\end{solution}
	\vfill
	\clearpage
\uplevel{Continuing from the previous page...}
	\part The polynomial $f(x) = x^6 - 3 x^5 + 6 x^4 - 11 x^3 + 12 x^2 + 3 x + 1$ happens to have $\sqrt[3]{2} + \omega$ as a root.  Use all the elements of the Galois group to find all the roots of the polynomial.  Briefly explain why this works.
	\begin{solution}
		The other roots will be the images of this number under each automorphism.  So for example, another one will be $\sqrt[3]{2}\omega + \omega^2$.  
	\end{solution}
	\vfill
	\part What can we now conclude about the field $\Q(\sqrt[3]{2}+\omega)$ and how it relates to $\Q(\sqrt[3]{2},\omega)$?  Justify your answer.
	\vfill

\end{parts}




\clearpage


\question[15] Think back over the material we have discussed so far in this course and identify one topic, theorem, proof, part of a proof, or even example that you either still do not understand or really struggled with before understanding.  Carefully explain the context and what your understanding of it is, and exactly where and why you are or were stuck.  Then briefly reflect (write about) what you have learned about mathematics and yourself through your struggle with this topic.

You should use \emph{at least} this full page for your response, using medium sized handwriting. 

\clearpage

\bonusquestion[15] Really fun BONUS! The polynomial $p(x) = x^7-1$ has a root $\alpha = e^{i2\pi/7}$.  This is a primitive 7th root of unity; the other imaginary roots are $\alpha^2, \alpha^3, \ldots, \alpha^6$.  On the other hand, recall that the group $U(7)$ is the group with set $\{1, 2,\ldots, 6\}$ under the operation multiplication mod 7.

Illustrate how $U(7)$ is isomorphic to the Galois group for $p(x)$.  Further, pick a subgroup and find its fixed field, and pick a different subfield (of the splitting field) and find its fixer.  Lots of partial credit available here.

	\begin{solution}
		The polynomial factors as $(x-1)(x^6+x^5+\cdots + x + 1)$, and $\alpha$ is a root of the degree 6 part.  This means $[\Q(\alpha):\Q)] = 6$.  The other five roots are all powers of $\alpha$, so in fact $\Q(\alpha)$ contains all the roots, so it is the splitting field.

		Each automorphism in the Galois group is determined by where it sends $\alpha$.  Let $\sigma_2(\alpha) = \alpha^2$ and $\sigma_3(\alpha) = \alpha^3$ (this is okay, since the roots of the minimal polynomial for $\alpha$ are $\alpha, \alpha^2, \alpha^3, \alpha^4,\alpha^5,\alpha^6$).  Now if we perform the group operation (composition) on these, we get $\sigma_2\circ \sigma_3(\alpha) = \sigma_2(\alpha^3) = (\alpha^3)^2 = \alpha^6$.

		The corresponding calculation in $\Z_7^*$ is $2\cdot 3 = 6$.
	\end{solution}


% \clearpage

	% \cleardoublepage
	% OLD QUESTIONS
	% \question[12] Let $\alpha$ be a real number for which $[\Q(\alpha):\Q] = 5$.  What does this tell you about\ldots
	% \begin{parts}
	% 	\part \ldots the number $\alpha$ in terms of polynomials?  Be as specific as possible.
	% 	\begin{solution}
	% 		That there is a degree 3 polynomial $p(x)$ that has $\alpha$ as a root.
	% 	\end{solution}
	% 	\vfill
	% 	\part \ldots about the field $\Q(\alpha)$?  What does the field look like (i.e., what does a basis look like)? 
	% 	\begin{solution}
	% 		A basis is $\{1, \alpha, \alpha^2\}$ (has size 3).  Every element in the field can be written $a + b \alpha + c \alpha^2$ for $a, b, c \in \Q$.
	% 	\end{solution}
	% 	\vfill
	% 	\part \ldots about the Galois group $G(\Q(\alpha)/\Q)$, if you know also that $\Q(\alpha)$ is a splitting field?  (You should say what sorts of things and how many of them are in the Galois group.)
	% 	\begin{solution}
	% 		The Galois group will contain three elements, each an automorphism of $\Q(\alpha)$ that leaves $\Q$ fixed.  In other words, $\Gal(\Q(\alpha)/\Q) \cong \Z_3$. 
	% 	\end{solution}
	% 	\vfill
		
	% 	\part \ldots about whether it is possible to construct a line segment of length $\alpha$ using a compass and straight edge?  Briefly explain.
	% 	\begin{solution}
	% 		$\alpha$ will NOT be a constructible number, since constructible numbers must exist in a field extension of $\Q$ that is degree $2^k$ for some $k$.
	% 	\end{solution}
	% 	\vfill
	% \end{parts}
	% \newpage
	% 
	% \question[12] Here is a useful fact about polynomials:
	% \[  \left(\frac{38 x^2}{429}+\frac{14 x}{143}+\frac{145}{429}\right)(x^4 - 3) + \left(\frac{-38 x^3}{429}+\frac{-14 x^2}{143}+\frac{-23 x}{143}- \frac{48}{143}\right)(x^3 + 2x - 6) = 1\]
	% \begin{parts}
	% \part In $\Q[x]$, does $x^3 + 2x - 6$ have an inverse?  Explain.
	% \vfill
	% \part Let $J = \langle x^4 - 3\rangle$.  In $\Q[x]/J$, find the inverse of $J+(x^3 + 2x - 6)$.  Explain your answer.
	% \vfill
	% \part In the field $Q(\sqrt[4]{3})$, find the inverse of $-6 + 2\sqrt[4]{3} + \sqrt[4]{3}^3$ written as a linear combination of the basis elements?  Explain your answer.  Your explanation should mention an isomorphism.
	% \vfill
	% 
	% \end{parts}
	% \newpage
	
	% \question[12] Consider the real number $\alpha = \sqrt{3 + \sqrt{3}}$.  
	% \begin{parts}
	% \part Is $\alpha$ algebraic over $\Q$?  Prove your answer.
	% \vfill
	% \part What does your answer to the previous question tell you about the basis for $\Q(\alpha)$ over $\Q$.
	% \vfill
	% \part Let $n$ be the degree of the field extension of $\Q(\alpha)$ over $\Q$.  Does this mean that $\alpha^n \in \Q$?  If not, what does it tell you about $\alpha^n$?
	% \vfill
	% \part Explain how you know that $\{1, \alpha, \alpha^2, \alpha^3\}$ is linearly independent over $\Q$.  \\(Recall that a set of vectors $\{v_1, v_2, \ldots, v_n\}$ is linearly \emph{dependent} if and only if there are scalars $c_1, c_2, \ldots, c_n$, not all zero, such that $c_1v_1 + c_2v_2 + \cdots + c_nv_n = 0$.)
	% \vfill
	% \end{parts}


	% \question[12] Consider the field $\Q(\sqrt[3]{5})$.  This field contains the number $\sqrt[3]{5}$.  What else does it contain?  For each item below, say whether the elements described are in $\Q(\sqrt[3]{5})$ and briefly justify your answer.  
	% \begin{parts}
	% 	\part $4 + \frac{2}{7}\sqrt[3]{5} - \sqrt[3]{5}^2$.
	% 	\begin{solution}
	% 		Yes, since $\Q(\sqrt[3]{5})$ is a field.
	% 	\end{solution}
	% 	\vfill
	% 	\part $\sqrt{1+\sqrt[3]{5}}$.
	% 	\begin{solution}
	% 		No.  In general, taking square roots of elements will not keep you in the field.  
	% 	\end{solution}
	% 	\vfill
	% 	\part $\sqrt{5}$.  Hint: tower rule.
	% 	\begin{solution}
	% 		No, this cannot be in there.  $\Q(\sqrt{5})$ is a degree 2 extension of $\Q$.  If it were a subfield, then its degree would need to divide the degree of the larger field extension, in this case 3.  But $2$ does not divide 3.
	% 	\end{solution}
	% 	\vfill
	% 	\part All the roots of $x^3 - 5$.
	% 	\begin{solution}
	% 		No.  There are some imaginary roots of the polynomial, which are not in this purely real field.
	% 	\end{solution}
	% 	\vfill
	% \end{parts}
	
	% \question[12] For each item below, say whether the statement is TRUE or FALSE and justify your answer.  If the statement is true, briefly explain why; if the statement is false, give a counterexample with brief explanation.
	
	% \begin{parts}
	% \part Every algebraic number is constructible.
	% \begin{solution}
	%   False.  For example, $\sqrt[3]{2}$ is algebraic (since it is the root of the polynomial $x^3 - 2$) but not constructible (since it does not live in a degree $2^k$ extension of $\Q$).
	% \end{solution}
	% \vfill
	% \part If $\alpha$ is an algebraic number, then $\Q(\alpha) \cong \Q[x]/\langle p(x) \rangle$ for some polynomial $p(x)$.
	% \begin{solution}
	%   True.  $p(x)$ will be the minimal polynomial for $\alpha$.  The isomorphism is given by the FHT, via the evaluation homomorphism (evaluating at $\alpha$).
	% \end{solution}
	% \vfill
	% \part If $\Q(\alpha)$ is a degree $2$ extension of $\Q$, then then $\alpha = \sqrt{c}$ for some $c \in \Q$.
	% \begin{solution}
	%   False.  For example, the polynomial $x^2 - 3x + 6$ is irreducible.  Let $\alpha$ be a root, so $\Q(\alpha)$ is a degree 2 extension of $\Q$.  However, if $\alpha = \sqrt{c}$, then since $\alpha^2 = 3\alpha - 6$ we would have $c = 9c - 36\alpha + 36$ which would say that $\alpha = \frac{8c+ 36}{36}$ which would make $\alpha$ rational.  
	% \end{solution}
	% \vfill
	% \part If $p(\alpha) = 0$, then $\Q(\alpha)$ is the splitting field for $p(x)$.
	% \begin{solution}
	%   False.  $p(x) = x^3 - 2$ has three roots, but $\Q(\sqrt[3]{2})$ does not contain all the roots of $p(x)$, so is not the splitting field for it.
	% \end{solution}
	% \vfill
	% \end{parts}
	
	
	% \newpage
	
	
	% \newpage
	% %Too hard?  Takehome?
	% \question Let $p(x) = (x^2 - 5)(x^3 - 7)$, and let $K$ be the splitting field of $p(x)$. 
	% \begin{parts}
	% \part[5] Prove that there is no automorphism of $K$ which sends $\sqrt{5}$ to $\sqrt[3]{7}$.  Show specifically what goes wrong using the homomorphism property.
	% \begin{solution}
	% 	If there were such an automorphism, say $\varphi$, then 
	% 	\[5 = \varphi(5) = \varphi(\sqrt{5}\sqrt{5}) = \varphi(\sqrt{5})\varphi{\sqrt{5}} = \sqrt[3]{7}^2\]
	% 	which is clearly false.
	% \end{solution}
	% \vfill
	% \part[5] Give an example of a non-trivial automorphism of $K$ and briefly explain how you know your example works.
	% \begin{solution}
	% 	We just need to send roots of irreducible polynomials to roots of other irreducible polynomials.  So let $\sigma:K \to K$ be be such that $\sigma(\sqrt{5}) = -\sqrt{5}$ and $\sigma(\sqrt[3]{7}) = \sqrt[3]{7}$.  
	% 	This is enough, since a basis for $K$ is $\{1, \sqrt{5}, \sqrt[3]{7}, \sqrt{5}\sqrt[3]{7}, \sqrt[3]{7}^2, \sqrt{5}\sqrt[3]{7}^2\}$, so saying where $\sqrt{5}$ and $\sqrt[3]{7}$ to is enough to specify the automorphism on every element, using the homomorphism property. 
	% \end{solution}
	% \vfill
	% \bonuspart[5] Bonus: Could $G(K/\Q)$ be isomorphic to $\Z_2\times \Z_3$?  Explain.
	% \begin{solution}
	% 	No.  The problem is the splitting field for $p(x)$ is not degree 6, since $p(x)$ has non-real roots, but $\Q(\sqrt{5},\sqrt[3]{7})$, already of degree 6, has only real number elements.
	% \end{solution}
	% \vfill
	% \end{parts} 
	% \clearpage 
	


% \question[4] Suppose $p(x)$ is an irreducible polynomial in $\Q[x]$ with root $c$.  Let $E$ be the splitting field of $p(x)$ (that is, $E$ contains all the roots of $p(x)$).  Prove that if $\varphi:E \to E$ is an automorphism, then $\varphi(c)$ is also a root of $p(x)$.




% \question[12] For each part below, either give an example and explain why the example works, or explain why no example exists.
% \begin{parts}
% % \part An ideal $J \subseteq \Q[x]$ and two \underline{non-zero} elements $a(x), b(x) \in \Q[x]$ such that\\ $(J+a(x))(J+b(x)) = J+0$.
% % \begin{solution}
% % 	We could have $p(x) = (x^2 - 3)(x^2-5)$ and $J = \langle p(x) \rangle$.  Then $a(x) = x^2-3$ and $b(x) = x^2 - 5$ are both non-zero elements of $\Q[x]$ (in fact, $J + a(x) \ne 0$ and $J+b(x) \ne 0$) but $J+p(x) = J+0$.  This particular example shows that $\Q[x]/J$ is not an integral domain.  
% % \end{solution}
% \part An algebraic number which cannot be constructed using a compass and straight edge.
% \begin{solution}
% 	For example, $a = \sqrt[3]{2}$ is such a number.  It is algebraic since it is the root of $x^3 - 2$, but not constructible since it is not in a degree $2^k$ extension of $\Q$.
% \end{solution}
% \part A degree 8 field extension  $\Q(a,b)$ of $\Q$ for which $b \in \Q(a)$.
% \begin{solution}
% For example, $\Q(\sqrt[8]{3}, \sqrt{3}) = \Q(\sqrt[4]{3})$ has degree 8 over $\Q$ since $\sqrt[8]{3}$ has minimum polynomial $x^8 - 3$ (which is irreducible by Eisenstein's criterion).
% \end{solution}
% \part A degree 7 field extension $\Q(a,b)$ of $\Q$ for which $b \notin \Q(a)$ and $a \notin \Q(b)$.
% \begin{solution}
% Impossible.  Suppose $\Q(a,b)$ had degree 7 over $\Q$.  Then since $[\Q(a,b):\Q] = [\Q(a,b):\Q(a)][\Q(a):\Q]$, and 7 is prime, we have that either $[\Q(a,b):\Q(a)] = 1$ or $[\Q(a):\Q] = 1$.  In the first case, this says that $\Q(a,b) = \Q(a)$ (in which case the extension is simple), in the second case this says that $\Q(a,b) = \Q(b)$ (again simple) since $\Q(a) = \Q$.
% \end{solution}
% \part A polynomial $p(x) \in \Q[x]$ of degree $n$, with a splitting field $E$ such that $[E:\Q] \ne n$.
% \begin{solution}
% 	Lots of examples of this.  For example, $x^3 - 2$ has a splitting field with degree 6.
% \end{solution}

% \end{parts}






%\bonusquestion[5] Bonus: There is a field of order 5 (containing exactly 5 elements), namely $\Z_5$.  Find a field of order 125.  Note that $\Z_{125}$ is not an example of this because $5\cdot 25 = 0$ in $\Z_{125}$.  Prove that your example works.
%\begin{solution}
%The idea is that if we can adjoin some element $\delta$ to get a degree 3 field extension of $\Z_5$, the basis will contain 3 elements, so we will have $\Z_5(\delta) = \{a + b\delta + c \delta^2 \st a,b,c\in\Z_5\}$.  Since there are 5 choices for each of $a$, $b$, and $c$, we will have $5^3 = 125$ different elements.  
%
%What should $\delta$ be?  Well the root of some irreducible degree 3 polynomial in $\Z_5[x]$.  Well, since we need a degree 3 polynomial, we can look for one that doesn't have any roots.  For example, $p(x) = x^3 +x+ 1$ is irreducible in $\Z_5$, so our field is $\Z_5[x]/\langle x^3 +x+ 1\rangle$.
%\end{solution}

\end{questions}




\end{document}


